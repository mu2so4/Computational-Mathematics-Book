\documentclass{article}

\begin{document}
\section{Вычислительные методы линейной алгебры}
Предмет исследования данного раздела -- это система линейных алгебраических
уравнений (СЛАУ)
\[
	\begin{cases}
		a_{11}x_1+a_{12}x_2+...+a_{1n}x_n=f_1, \\
		a_{21}x_1+a_{22}x_2+...+a_{2n}x_n=f_2, \\
		... \\
		a_{n1}x_1+a_{n2}x_2+...+a_{nn}x_n=f_n. \\
	\end{cases}
\]

В матричной форме также можно записать как
\[Ax=f.\]

Если определитель матрицы $A$ отличен от нуля, то решение существует, и притом
только одно. Казалось, почему нельзя просто применить для решения метод Крамера
или обращение матрицы? Как и всегда: трудоёмкость этих методов быстро растёт с
ростом количества уравнений системы.

Например, для решения СЛАУ методом Крамера необходимо посчитать $n+1$
определитель матрицы, а вычислительная сложность этой операции составляет
$O(n!)$. В итоге, общая вычислительная мощность составляет примерно $O((n+1)!)$!
Для понимания масштабов трагедии: если система состоит из 20 уравнений, на
одну атомарную операцию уходит $10^{-12}$ секунд, и вычисления не
распараллеливаются, то дл решения потребуется 1.62 года непрерывных вычислений,
и притом без перебоев.

Поэтому системы решают иначе. В этом разделе будут рассмотрены методы Гаусса и
итерационные методы.

\end{document}
