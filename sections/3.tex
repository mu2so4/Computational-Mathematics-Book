\documentclass[../main.tex]{subfile}

\begin{document}

\section{Численные методы интегрирования}

Как мы находили определённый интеграл от функции $f(x)$ на отрезке $[a,b]$? Мы
всевозможными способами искали первообразную $F(x)$, а затем по формуле
Ньютона-Лейбница $\int_a^bf(x)dx=F(b)-F(a)$ вычисляли его значение. И эта
формула работала.

Проблемы начинаются тогда, когда у функции не получается найти первообразную
или она выглядит слишком сложно или страшно, чтобы в неё что-то там подставлять. Например,
известными нам методами невозможно найти первообразные функций $f(x)=e^{-x^2}$
и $f(x)=x^x$.

Это значит, что отталкиваться придётся только от исходной функции $f(x)$. О том,
как приближённо найти интеграл без первообразной, и будет рассказано в этой главе.

\begin{define}
	\textbf{Задача численного интегрирования} -- задача нахождения
	определённого интеграла функции без использования её первообразной и
	формулы Ньютона-Лейбница.
\end{define}

Наиболее часто для численного интегрирования используется следующая формула.

\begin{define}
	\textbf{Квадратурная формула} -- формула вида
	\[\int_a^bf(x)dx\approx\sum_{k=0}^n c_kf(x_k),\]
	где $c_k$ -- \textbf{весовые коэффициенты}, а $x_k\in[a,b]$ --
	\textbf{узлы интегрирования}.
\end{define}

\begin{define}
	\textbf{Кубатурная формула} -- формула вида
	\[\int_{D}f(x)dx\approx\sum_{k=0}^n c_k f(x_k),\quad x_k\in D,\;D
	\subseteq\mathbb R^m,\;m\ge 2.\]
\end{define}

\begin{define}
	\textbf{Сетка} -- это совокупность узлов $\{x_1,...,x_n\}$ квадратурной
	или кубатурной формулы.
\end{define}

\begin{define}
	\textbf{Погрешность квадратурной формулы} или её \textbf{остаточный
	член} -- это разность
	\[R_{[a,b]}(f)=\int_a^b f(x)dx - \sum_{k=0}^n c_kf(x_k).\]
\end{define}

\begin{define}
	Квадратурная формула называется \textbf{точной} для функции $f$ или
	класса $\mathcal F\ni f$ на интервале $[a,b]$, если верно $R_{[a,b]}(f)=0$.
\end{define}

То, насколько широк класс $\mathcal F$, на которых квадратурная формула точна,
может говорить о точности квадратурной формулы в целом. Часто в качестве класса
''пробных'' функций $\mathcal F$ берут алгебраические полиномы.

\begin{define}
	\textbf{Алгебраическая степень точности} квадратурной формулы --
	наибольшая степень алгебраического полинома, для которого эта формула
	точна на заданном отрезке $[a,b]$.
\end{define}

Конечно, для нас предпочтительней формулы с б\'{о}льшей алгебраической степенью
точности.

\end{document}
