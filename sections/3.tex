\documentclass[../main.tex]{subfile}

\begin{document}

\section{Численные методы интегрирования}

Как мы находили определённый интеграл от функции $f(x)$ на отрезке $[a,b]$? Мы
всевозможными способами искали первообразную $F(x)$, а затем по формуле
Ньютона-Лейбница $\int_a^bf(x)dx=F(b)-F(a)$ вычисляли его значение. И эта
формула работала.

Проблемы начинаются тогда, когда у функции не получается найти первообразную
или она выглядит слишком сложно или страшно, чтобы в неё что-то там подставлять. Например,
известными нам методами невозможно найти первообразные функций $f(x)=e^{-x^2}$
и $f(x)=x^x$.

Это значит, что отталкиваться придётся только от исходной функции $f(x)$. О том,
как приближённо найти интеграл без первообразной, и будет рассказано в этой главе.

\begin{define}
	\textbf{Задача численного интегрирования} -- задача нахождения
	определённого интеграла функции без использования её первообразной и
	формулы Ньютона-Лейбница.
\end{define}

Наиболее часто для численного интегрирования используется следующая формула.

\begin{define}
	\textbf{Квадратурная формула} -- формула вида
	\[\int_a^bf(x)dx\approx\sum_{k=0}^n c_kf(x_k),\]
	где $c_k$ -- \textbf{весовые коэффициенты}, а $x_k\in[a,b]$ --
	\textbf{узлы интегрирования}.
\end{define}

\begin{define}
	\textbf{Кубатурная формула} -- формула вида
	\[\int_{D}f(x)dx\approx\sum_{k=0}^n c_k f(x_k),\quad x_k\in D,\;D
	\subseteq\mathbb R^m,\;m\ge 2.\]
\end{define}

\end{document}
