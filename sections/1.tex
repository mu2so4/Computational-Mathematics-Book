\documentclass[../main.tex]{subfile}

\begin{document}

\section{Численное решение\\ алгебраических уравнений}

Предположим, что нам нужно решить в действительных числах уравнение
$y(x)=0$. Если это просто абстрактное уравнение, то нет единого
алгоритма, как его решить. Первое, что можно сделать -- это локализовать
корни, но это можно сделать лишь по виду функции.

\subsection{Метод бисекции}

Простейший метод отыскания нуля функции -- это метод бисекции, или метод
деления отрезка пополам. Для него нужна лишь совсем небольшая математическая
база.

\begin{algorithm}[метод бисекции]\label{eq:bisection_method}
	Необходимо найти ноль функции $f(x)$ на отрезке $[a,b]$ таком, что:
	\begin{enumerate}[noitemsep, nolistsep]
		\item $f(x)\in C([a,b])$;
		\item $f(a)f(b)<0$, то есть на концах отрезка функция
			принимает значения, противоположные по знаку.
	\end{enumerate}

	Организуем систему вложенных отрезков $[a_n, b_n]$ так, что\\
	$[a_1, b_1]=[a,b]$, а следующие отрезки сформированы следующим образом:
	если $c_k=\frac{a_k+b_k}{2}$, то:
	\begin{itemize}[noitemsep, nolistsep]
		\item Если $f(c_k)=0$, то мы нашли корень, можно не продолжать;
		\item В противном случае, мы берём тот конец отрезка, функция от
			которой не совпала по знаку с $f(c_k)$, и формируем из
			него и точки $c_k$ отрезок $[a_{k+1},b_{k+1}]$.
	\end{itemize}

	Условие выхода: $|f(c_n)|<\varepsilon$, где $\varepsilon>0$ -- некоторая
	погрешность.

	Если данный алгоритм усилить монотонным возрастанием или убыванием на
	$[a,b]$, то ноль на нём будет единственный.
\end{algorithm}

Корректность данного алгоритма напрямую следует из теоремы Больцано-Коши, которая
и доказывается методом бисекции в сочетании с теоремой о двух милиционерах.

\begin{example}
	Пусть нужно решить уравнение
	\[f(x)=x^3-6x^2+8x+2=0.\]

	Функция непрерывна на всей вещественной оси. Значит, с локализацией
	корней проблем быть не должно.
	\newline

	\begin{tabular}{ |c|c|c|c|c| }
		\hline
		$x$		& 0	& 1	& 2	& 3 \\
		\hline
		Знак $f(x)$ 	& $+$	& $+$	& $+$	& $-$ \\
		\hline
	\end{tabular}
	\newline

	Тогда $\exists c\in[2,3]: f(c)=0$. Найдём его методом бисекции:
	\newline

	\begin{tabular}{*{4}{|c}|}
		\hline
		$x$ &	2.5	& 2.75	& 2.625 \\
		\hline
		$f(x)$ &0.125	& -0.578&-0.226 \\
		\hline
	\end{tabular}\leavevmode\newline

	Представим метод бисекции графически:
	\newline

	\subfile{graphs/1.1-cubical}

\end{example}


\subsection{Метод простой итерации}

Метод бисекции гарантированно даёт нам результат. Но нам бы хотелось иметь
более быструю сходимость. В этом нам могут помочь итерационные методы.

\begin{define}
	\textbf{Итерацией} называется многократное применение одной и той же
	функции $f$ к числу. Пусть задано начальное число $x_0$, тогда:
	\begin{itemize}[noitemsep, nolistsep]
		\item $x_1=f(x_0)$,
		\item $x_2=f(x_1)=f(f(x_0))$,\\
		...
		\item $x_n=f(x_{n-1})=\underset{n}{\underbrace{f(f(...f}}
			(x_0)...))$.
	\end{itemize}

	\textbf{Последовательность} $\{x_n\}$, образованная таким образом,
	называется \textbf{итерационной} с базой $x_0$ и функцией итерации $f(x)$.
\end{define}

\begin{algorithm}[метод простой итерации]\label{eq:fp_iteration}
	Пусть нужно решить уравнение $y(x)=0$. Сделаем подстановку
	$g(x)=x+\tau y(x)$, где $\tau$ -- некоторая положительная константа.
	Таким образом, теперь нужно решить уравнение $g(x)=x$, то есть вместо
	нулей функции мы ищем неподвижные точки.

	Пусть $x_0$ -- некоторое начальное приближение. Построим итерационную
	последовательность с базой $x_0$ и функцией итерации $g(x)$. Если
	параметр $\tau$ подобран правильно, итерационная последовательность
	сойдётся к неподвижной точке. При заданной погрешности $\varepsilon$
	завершаем итерацию при $|g(x_n)-x_n|<\varepsilon$.
\end{algorithm}

Как нужно подобрать $\tau$, чтобы последовательность сошлась? Однозначного
ответа нет. Однако тут частично выручает та теория, которую мы узнали на ДГМА.
Дело идёт о так называемых сжимающих операторах.

Для начала не помешает освежить в своей памяти некоторые определения.

\begin{define}
	\textbf{Метрическим пространством} называется пространство $M$, в
	котором определена операция метрики $\rho: M^2 \rightarrow \mathbb R $
	такая, что $\forall x, y, z \in M$ верно
	\begin{enumerate}[noitemsep, nolistsep]
		\item $\rho(x,y)\ge 0$, причём $\rho(x,y) = 0 \Leftrightarrow x=y$;
		\item $\rho(x,y)=\rho(y,x)$;
		\item $\rho(x,y)\le \rho(x,z) + \rho(z,y)$.
	\end{enumerate}

	Если дополнительно верно, что каждая фундаментальная последовательность
	$M$ сходится к элементу, принадлежащему ему, то $M$ --
	\textbf{полное метрическое пространство}.
\end{define}

\begin{define}
	\textbf{Сжимающим оператором} на метрическом пространстве $M$ с метрикой
	$\rho$ называется оператор $f: M \rightarrow M$ такой, что\\
	$\exists q \in(0;1):\forall x,y \in M\;\; \rho(f(x),f(y)) \le q \rho(x,y)$.

	Число $q$ назовём \textbf{коэффициентом сжатия}.
\end{define}

\begin{theorem}[о сжимающем операторе]\label{eq:banach_fp_theorem}
	Пусть $(M,\rho)$ -- полное метрическое пространство, $f$ -- сжимающий
	оператор на $M$ с коэффициентом сжатия $q$. Тогда:
	\begin{enumerate}[noitemsep, nolistsep]
		\item $\exists!\;\widetilde{x}\in M: f(\widetilde{x})=\widetilde{x}$;
		\item Любая последовательность $\{x_n\}$ такая, что $x_0 \in M$ --
			произвольный, $x_{k+1}=f(x_k)$, сходится к $\widetilde{x}$;
		\item Для всякой такой последовательности верно\\
			$\rho(x_n, \widetilde{x}) \le q ^ n \rho(x_0, \widetilde{x})$.
	\end{enumerate}
\end{theorem}

Данная теорема была доказана в курсе ''Дополнительные главы математического
анализа''. В литературе эта теорема также известна как ''Теорема Банаха о
неподвижной точке''.

Всё это нужно было для того, чтобы доказать красивую, но практически бесполезную
теорему.

\begin{theorem}[о сходимости итерационного процесса]
	Пусть нам необходимо найти неподвижную точку функции $g(x)$. Зададим
	в качестве полного метрического пространства отрезок $G=\{x: |x-a| \le r\}$,
	где $a$ и $r$ -- параметры отрезка. Для того чтобы итерационный процесс
	$x_{k+1}=g(x_k)$ сошёлся, достаточно, чтобы:
	\begin{enumerate}[noitemsep, nolistsep]
		\item $\forall x,y\in G\;|g(x)-g(y)|\le q|x-y|$, где $q\in(0,1)$ --
			некоторая константа, то есть функция непрерывна по Липшицу;
		\item $|g(a)-a|\le(1-q)r$.
	\end{enumerate}
\end{theorem}
\newpage

\begin{proof}
	Нам всего лишь нужно доказать, что $g(x)$ замкнута на $G$.
	\begin{multline*}
		\forall x\in G:|g(x)-a|=|g(x)-g(a)+g(a)-a|\le \\
		\le|g(x)-g(a)|+|g(a)-a|.
	\end{multline*}

	Применив условия 1 и 2 к первому и второму модулю соответственно, получаем
	\begin{align*}
		|g(x)-g(a)|+|g(a)-a|\le q|x-a|+(1-q)r\le qr+r-qr=r.
	\end{align*}

	Следовательно, $\forall x\in G: |g(x)-a|\le r\Rightarrow\forall x\in
	G\;\;g(x)\in G$, что и даёт нам замкнутость $g(x)$ на этом отрезке.
	Условие 1 и замкнутость даёт нам право считать $g(x)$ сжимающим
	отображением, поэтому тут применима теорема Банаха о неподвижной точке
	\eqref{eq:banach_fp_theorem} -- итерационный процесс сходится.
\end{proof}

\begin{corollary}\label{eq:derivative_condition}
	Пусть функция $f(x)\in C[a,b]=G$, причём $\forall c\in G$ $|f'(c)|<1$,
	а также она замкнута на $G$. Тогда $\exists x\in G: f(x)=x$.
\end{corollary}

\beginproof

	Так как модуль производной ограничен сверху, будем считать, что
	$|f'(x)|\le q<1$.

	По теореме Лагранжа о среднем значении, $\forall x,y\in G$ \\
	\[\exists z\in[x,y]: f(y)-f(x)=f'(z)(y-x).\]

	Добавим модули: $|f(y)-f(x)|=|f'(z)||y-x|\le q|y-x|$, то есть имеем
	непрерывность по Липшицу, и тогда неподвижная точка на $G$ существует.
	\qed

\begin{example}[графическая интерпретация метода простой итерации]
	Нужно решить уравнение \(e^x=x+2.\) Как и в предыдущем примере, с
	непрерывностью у данной функции проблем нет.
	\newpage

	Отобразим функции $g(x)=e^x$ и $h(x)=x+2$ на графике:

	{\makeatletter
	\let\par\@@par
	\par\parshape0
	\everypar{}
	\begin{wrapfigure}{l}{0.45\textwidth}
		\subfile{graphs/1.2-fixed-point}
	\end{wrapfigure}

	Видно, что у уравнения имеются 2 решения. Обозначим их как
	$\widetilde{x_1}$ и $\widetilde{x_2}$. Запишем уравнение так, будто мы
	ищем нули функции: $f(x) = e^x-x-2$. Теперь локализуем корни:\\\\
	\begin{tabular}{ *{6}{|c}| }
		\hline
		$x$		& -2	& -1	& 0	& 1	& 2 \\
		\hline
		Знак $f(x)$ 	& $+$	& $-$	& $-$	& $-$ 	& $+$\\
		\hline
	\end{tabular}
	\\

	$\widetilde{x_1}\in[-2,-1],\;\widetilde{x_2}\in[1,2]$.

	Теперь запишем уравнение через поиск неподвижной точки, то есть как
	$u(x)=e^x-2$.
	\par}


	Найдём сначала $\widetilde{x_1}$. За начальную точку возьмём $x_0=-1.5$ --
	середину отрезка. Поехали:

	\begin{tabular}{*{4}{|c}|}
		\hline
		$k$		& 0	& 1	& 2	\\
		\hline
		$x_k$		&-1.5000&-1.7769&-1.8308\\
		\hline
		$|u(x_k)-x_k|$	&0.2769	&0.0539	&0.0089	\\
		\hline
	\end{tabular}
	\newline

	Итерационная последовательность очень быстро сходится к $\widetilde{x_1}$.
	Не менее интересно будет посмотреть на это на графике:
	\newline

	\subfile{graphs/1.2-first-point}

	\newpage

	Теперь попробуем найти $x_2$, который где-то на отрезке $[1,2]$. Как и
	в предыдущем случае, за $x_0$ возьмём середину отрезка, то есть 1.5:
	\begin{itemize}[noitemsep, nolistsep]
		\item $x_1=u(1.5000)\approx 2.4817$, $|u(x_0)-x_0|\approx 0.9817$;
		\item $x_2=u(2.4817)\approx 9.9616$, $|u(x_1)-x_1|\approx 7.4799$;
	\item $x_3=u(9.9616)\approx 21195$ и т. д.
	\end{itemize}

	Вот тут нам не повезло: итерационная последовательность разошлась.
	На это тоже можно посмотреть на графике:
	\newline
	\subfile{graphs/1.2-second-point}

	Во втором случае расхождение последовательности произошло из-за того,
	что производная $\forall x\ge 1: |u'(x)|>1$.
\end{example}

Однако невыполнение условий теоремы ещё не означает, что итерационная
последовательность не сойдётся к неподвижной точке.

\begin{example}\label{eq:ex_ex}
	Необходимо найти неподвижные точки функции \[f(x)=-e^{2x}+4.5e^x-3.\]

	Предположим, мы уже локализовали одну неподвижную точку на отрезке
	$[-1,0]$. Найдём значения производной $f'(x)=-2e^{2x}+4.5e^x$ в крайних
	точках:
	\[f'(0)=2.5,\quad f'(-1)=-\frac{2}{e^2}+\frac{4.5}{e}>-\frac{2}{4}+
	\frac{4.5}{3}=1.\]

	То есть, по следствию \eqref{eq:derivative_condition}, итерационная
	последовательность должна разойтись. Проверим это: возьмём $x_0=-0.5$
	и начнём строить итерационную последовательность:

	\begin{table}[h]
		\centering
		\resizebox{\columnwidth}{!}{\begin{tabular}{*{10}{|c}|}
			\hline
			$k$		& 0	& 1	& 2	& 3	& 4
				& 5	& 6	&7	&8	\\
			\hline
			$x_k$		&-0.6385&-0.9025&-1.3395&-1.8897&-2.3428
				&-2.5770&-2.6638&-2.6913&-2.6995\\
			\hline
			$|f(x_k)-x_k|$	& 0.1385& 0.2640& 0.4370& 0.5502& 0.4531
				& 0.2342& 0.0868& 0.0275& 0.0082\\
			\hline
		\end{tabular}}
	\end{table}
	\subfile{graphs/1.2-derivative-example}
	С другой стороны, итерационная последовательность сошлась к числу не из
	нашего отрезка.
\end{example}

\subsection{Метод Ньютона}

Одним из самых эффективных численных методов решения нелинейных уравнений на
отрезке $[a,b]$ вида $f(x)=0$ является метод Ньютона, он же метод касательных.
Здесь будет испозьзована линеаризация уравнения, которая сводит решение
нелинейной задачи к решению последовательности линейных задач.

\begin{algorithm}[метод Ньютона]\label{eq:newton_iteration}
	Пусть для уравнения $f(x)=0$ построено приближение $x_k$ к корню
	$\widetilde{x}$. Представим функцию $f(x)$ в окрестности точки $x_k$ в
	виде ряда Тейлора:
	\[f(x)=f(x_k)+f'(x_k)(x-x_k)+\frac{f''(x_k)}{2!}(x-x_k)^2+...\;.\]

	Возьмём от уравнения только линейную часть и положим \\
	$f(x)=0$. Тогда мы имеем:
	\[f(x_k)+f'(x_k)(x-x_k)=0.\]

	Положив решение уравнения относительно $x$ новым приближением к
	$\widetilde{x}$ и обозначением $x_{k+1}$, мы получили следующую формулу:
	\[\boxed{x_{k+1}=x_k-\frac{f(x_k)}{f'(x_k)}}\]

	Условие завершения алгоритма: $|f(x_n)|<\varepsilon$.
\end{algorithm}

Теперь не помешало бы определить область задач, где метод Ньютона применим.

\begin{theorem}[условие сходимости метода Ньютона]\label{eq:newton_cond}
	Пусть $f(x)$ на отрезке $[a,b]=G$ обладает следующими свойствами:
	\begin{enumerate}[noitemsep, nolistsep]
		\item $f(x)\in C^2(G)$;
		\item $f(a)f(b)<0$;
		\item $f'(x)$ и $f''(x)$ отличны от нуля и знакопостоянны на $G$;
		\item Для начального приближения $x_0\in G$ верно
			$f(x_0)f''(x_0)>0$.
	\end{enumerate}

	Тогда у функции существует единственный ноль $\widetilde{x}$ на $G$, а
	итерационная последовательность $\{x_n\}$, построенная по методу
	Ньютона, \underline{монотонно} сходится к нему.
\end{theorem}

\begin{proof}
	Из условий 2 и 3 по теореме Больцано-Коши мы сразу получаем существование
	и единственность нуля функции $\widetilde{x}\in G$.

	Далее, для определённости будем считать, что
	\[f(a)<0, f(b)>0,\;\forall x\in G f'(x)>0, f''(x)>0.\]

	Доказательство для других случаев аналогичное.

	В рассматриваемом случае, из условия 4 $f(x_0)>0$, можем взять конкретно
	$x_0=b$. Теперь методом математической индукции докажем, что
	$\forall n\in \mathbb N\;\;x_n>\widetilde{x}$.

	Для $x_0=b$ это тривиально, так как функция возрастающая.

	Теперь положим, что верно $x_k>\widetilde{x}$. Покажем, что для $k+1$
	это тоже верно.

	Положив, что $z_k\in[\widetilde{x},x_k]$, разложим в ряд Тейлора:
	\[0=f(\widetilde{x})=f(x_k)+f'(x_k)(\widetilde{x}-x_k)+
		\underset{>0}{\underbrace{\frac{f''(z_k)}
		{2}(\widetilde{x}-x_k)^2}}.\]

	Откинув последнее слагаемое, получаем:
	\[f(x_k)+f'(x_k)(\widetilde{x}-x_k)<0\Rightarrow
		\widetilde{x}<\underset{x_{k+1}}{
			\underbrace{x_k-\frac{f(x_k)}{f'(x_k)}}}.\]

	Из этого же неравенства вытекает $x_{k+1}<x_k$. Тогда последовательность
	$\{x_n\}$ имеет предел $\overline{x}$, поскольку она монотонно убывает и
	ограничена снизу. Перейдём к пределу в формуле метода Ньютона
	\eqref{eq:newton_iteration}:
	\[\overline{x}=\overline{x}-\frac{f(\overline{x})}{f'(\overline{x})}
		\Rightarrow f(\overline{x})=0\Rightarrow \overline{x}=
		\widetilde{x}.\]
\end{proof}

На словах нельзя утверждать, что метод Ньютона очень эффективен. Чтобы проверить
его на деле, нужно узнать, как быстро уменьшается отклонение от искомого нуля
функции.

Сначала дадим общую оценку.

\begin{lemma}[общая оценка погрешности]\label{eq:general_error}
	Пусть $\widetilde{x}$ и $\overline{x}$ -- точный и приближённый корни
	уравнения $f(x)=0$ на отрезке $[a,b]$ соответственно, сама функция
	$f(x)$ такая, что \\ $\forall x\in[a,b]\;\;|f'(x)|\ge m_1>0$. Тогда
	верно неравенство
	\[\boxed{|\overline{x}-\widetilde{x}|\le\frac{|f(\overline{x})|}
	{m_1}}\]
\end{lemma}

\begin{proof}
	Из существования производной следует непрерывность функции, тогда
	применима теорема Лагранжа о среднем:
	\[f(\overline{x})-f(\widetilde{x})=f'(z)(\overline{x}-\widetilde{x}),\]
	где $z\in[\overline{x},\widetilde{x}]\subseteq[a,b]$. Добавим модулей:
	\[|f(\overline{x})-\underset{=0}{\underbrace{f(\widetilde{x})}}|=|f'(z)|
		|\overline{x}-\widetilde{x}|\ge m_1|\overline{x}-\widetilde{x}|
		\Rightarrow |\overline{x}-\widetilde{x}|\le
		\frac{|f(\overline{x})|}{m_1}.\]
\end{proof}

Так мы можем проверить погрешность для каждого члена последовательности. Тем не
менее, мы можем усилить оценку в случае именно метода Ньютона.

\begin{theorem}[об оценке погрешности метода Ньютона]
	Если для функции $f(x)$ на отрезке $[a,b]=G$, с нулём функции
	$\widetilde{x}\in G$ и стартовом значении $x_0\in G$ применим метод
	Ньютона, то для любого члена итерационной последовательности $\{x_n\}$
	верно неравенство
	\[\boxed{|\widetilde{x}-x_{n+1}|\le \frac{M_2}{2m_1}
	(\widetilde{x}-x_n)^2},\]
	где $m_1=\underset{x\in G}{min}|f'(x)|$, а $M_2=\underset{x\in G}{max}
	|f''(x)|$.
\end{theorem}

\beginproof

	Представим функцию $f(x)$ в окрестности $x_{n-1}$ в виде ряда Тейлора:
	\[f(x_n)=f(x_{n-1})+f'(x_{n-1})(x_n-x_{n-1})+\frac{f''(z_{n-1})}{2}
	(x_n-x_{n-1})^2,\]
	где $z_{n-1}\in[x_{n-1},x_n]$. В силу определения $x_n$
	\eqref{eq:newton_iteration},
	\[f(x_{n-1})+f'(x_{n-1})(x_n-x_{n-1})=0.\]

	Тогда после добавления модулей имеем:
	\[|f(x_n)|=\frac{|f''(z_{n-1})|}{2}(x_n-x_{n-1})^2\le \frac{M_2}{2}
	(x_n-x_{n-1})^2,\]

	И с учётом общей оценки погрешности \eqref{eq:general_error} получаем
	необходимое неравенство.\qed

\begin{example}
	Попробуем снова найти нули функции
	\[f(x)=-e^{2x}+4.5e^x-x-3\]
	из примера \eqref{eq:ex_ex}, в этот раз методом Ньютона. Чтобы не
	растягивать пример, найдём корень лишь из того же отрезка $[-1,0]$.

	Первая и вторая производные положительны на отрезке $[-1,0]$. Тогда
	мы положим $x_0=0$, чтобы выполнялось достаточное условие сходимости
	\eqref{eq:newton_cond}.
	\subfile{graphs/1.3-newton-method}

	Точка $x_2$ на графике практически неотличима от $\widetilde{x}$.
\end{example}

\subsection{Модификации метода Ньютона}

Всё это работает, пока у нас корень только одинарной кратности. При большей
кратности метод Ньютона \eqref{eq:newton_iteration} перестаёт работать, потому
что производная при приближении к нулю функции тоже стремится к нулю, а в
формуле она в знаменателе. Надо как-то изменить формулу.

\begin{algorithm}[метод одной касательной]
	Пусть для уравнения $f(x)=0$ построено начальное приближение $x_0$, для
	которой существует и конечна производная $f'(x_0)$. Тогда следующие
	члены итерационной последовательности $\{x_n\}$ вычисляются по формуле
	\[\boxed{x_{k+1}=x_k-\frac{f(x_k)}{f'(x_0)}}.\]

	Условие завершение алгоритма: $|f(x_n)|<\varepsilon$.

	Метод используется тогда, когда требуется сократить число вычислений
	производной функции. Он является частным случаем метода простой
	итерации \eqref{eq:fp_iteration}.
\end{algorithm}

\begin{algorithm}[метод хорд]
	Пусть для уравнения $f(x)=0$ мы знаем, $f(a)f(b)<0$, а сама функция
	непрерывна. Возьмём $x_0=a,\;x_1=b$. Проведём через них прямую и отметим
	её точку пересечения с осью абсцисс $x_2$. Возьмём среди предыдущих
	точек ту, чья функция другого знака относительно $f(x_2)$. Повтореям так
	до тех пор, пока не получим $|f(x_n)|<\varepsilon$.

	Построим прямую $y=kx+b$, которая соединяет точки $x_{n-1}$ и $x_n$.
	\[k=\frac{\Delta y}{\Delta x}=\frac{f(x_n)-f(x_{n-1})}{x_n-x_{n-1}}.\]

	Зная, что точка $(x_n,f(x_n))$ лежит на этой прямой, имеем:
	\[f(x_n)=kx_n+b\Rightarrow b=f(x_n)-kx_n\].
	Окончательно решим уравнение:
	\[kx_{n+1}+b=0\Rightarrow x_{n+1}=-\frac{b}{k}=-\frac{f(x_n)-kx_n}{k},\]
	\[\boxed{x_{n+1}=x_n-\frac{x_n-x_{n-1}}{f(x_n)-f(x_{n-1})}f(x_n)}\]
\end{algorithm}

\begin{example}
	Найдём нули функции $f(x)=2\sin x-x$. Да, мы знаем, что $x=0$ подходит,
	но это не единственный корень. Другой ноль функции мы локализовали на
	отрезке $[1,2]$.
	\subfile{graphs/1.4-chorde-method}
\end{example}

\end{document}
