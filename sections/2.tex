\documentclass[../main.tex]{subfile}

\begin{document}

\section{Алгебраические методы интерполирования}
Задача интерполирования функции состоит в том, чтобы по известным её значениям
в некоторых точках определить её значения в остальных точках области задания.
Такая задача возникает, например, когда по результатам измерения некоторой
физической величины в одних точках требуется определить её значения в других
точках или когда в целях ускорения вычислений желательно приблизить заданную
функцию более лёгкой в вычислении. Как правило, интерполируют полиномами.

\begin{define}
	Алгебраический полином $P_m(x)=\sum_{k=0}^{m}a_kx^k$ называется
	\textbf{интерполяционным} для функции $f(x)$, заданной на отрезке
	$[a,b]$ по её значениям $f(x_i)$ в $n+1$ попарно различных точках
	$x_i\in[a,b]$ (\textbf{узлах интерполяции}), если
	\[\forall i\in\{0,1,...,n\}\;P_m(x_i)=f(x_i).\]

	\textbf{Задача алгебраической интерполяции} -- построить алгебраический
	полином, который удовлетворял бы данному условию.
\end{define}

Первый вопрос: а какие условия гарантируют существование и единственность
интерполяционного многочлена? Для $m+1$ неизвестного члена полинома мы имеем
$n+1$ условие. Единственность решения математических конечномерных задач
обычно обеспечивается равенством числа неизвестных числу условий. В противном
случае, мы можем либо получить несколько решений, либо не получить их вовсе.

\begin{theorem}[о единственности]
	Задача алгебраической интерполяции при $n=m$ имеет единственное решение.
\end{theorem}

\beginproof

	Запишем систему в матричном виде:
	\[
	\begin{pmatrix}
		1	& x_0	& x_0^2	& ...	& x_0^n \\
		1	& x_1	& x_1^2	& ...	& x_1^n \\
		...	& ...	& ...	&\ddots	& ...	\\
		1	& x_n	& x_n^2	& ...	& x_n^n \\
	\end{pmatrix}
	\begin{pmatrix}
		a_0 \\
		a_1 \\
		... \\
		a_n \\
	\end{pmatrix}
	=
	\begin{pmatrix}
		f(x_0)	\\
		f(x_1)	\\
		...	\\
		f(x_n)	\\
	\end{pmatrix}
	\]

	Определитель этой матрицы -- определитель Вандермонда -- не равен нулю,
	так как $x_i\ne x_j$ при $i\ne j$. Это необходимо и достаточно для
	существования и единственности решения.\qed
\end{document}
