\documentclass[../main.tex]{subfile}

\begin{document}

\section{Алгебраические методы интерполирования}
Задача интерполирования функции состоит в том, чтобы по известным её значениям
в некоторых точках определить её значения в остальных точках области задания.
Такая задача возникает, например, когда по результатам измерения некоторой
физической величины в одних точках требуется определить её значения в других
точках или когда в целях ускорения вычислений желательно приблизить заданную
функцию более лёгкой в вычислении. Как правило, интерполируют полиномами.

\begin{define}
	Алгебраический полином $P_m(x)=\sum_{k=0}^{m}a_kx^k$ называется
	\textbf{интерполяционным} для функции $f(x)$, заданной на отрезке
	$[a,b]$ по её значениям $f(x_i)$ в $n+1$ попарно различных точках
	$x_i\in[a,b]$ (\textbf{узлах интерполяции}), если
	\[\forall i\in\{0,1,...,n\}\;P_m(x_i)=f(x_i).\]

	\textbf{Задача алгебраической интерполяции} -- построить алгебраический
	полином, который удовлетворял бы данному условию.
\end{define}
\end{document}
