\documentclass[../main.tex]{subfile}

\begin{document}

\section{Алгебраические методы интерполирования}
Задача интерполирования функции состоит в том, чтобы по известным её значениям
в некоторых точках определить её значения в остальных точках области задания.
Такая задача возникает, например, когда по результатам измерения некоторой
физической величины в одних точках требуется определить её значения в других
точках или когда в целях ускорения вычислений желательно приблизить заданную
функцию более лёгкой в вычислении. Как правило, интерполируют полиномами.

\begin{define}
	Алгебраический полином $P_m(x)=\sum_{k=0}^{m}a_kx^k$ называется
	\textbf{интерполяционным} для функции $f(x)$, заданной на отрезке
	$[a,b]$ по её значениям $f(x_i)$ в $n+1$ попарно различных точках
	$x_i\in[a,b]$ (\textbf{узлах интерполяции}), если
	\[\forall i\in\{0,1,...,n\}\;P_m(x_i)=f(x_i).\]

	\textbf{Задача алгебраической интерполяции} -- построить алгебраический
	полином, который удовлетворял бы данному условию. Далее будем писать
	просто ЗАИ.
\end{define}

Первый вопрос: а какие условия гарантируют существование и единственность
интерполяционного многочлена? Для $m+1$ неизвестного члена полинома мы имеем
$n+1$ условие. Единственность решения математических конечномерных задач
обычно обеспечивается равенством числа неизвестных числу условий. В противном
случае, мы можем либо получить несколько решений, либо не получить их вовсе.

\begin{theorem}[о существовании и единственности интерполяционного многочлена]
\label{eq:polynominal_theorem}
	ЗАИ при $n=m$ имеет единственное решение.
\end{theorem}

\beginproof

	Запишем систему в матричном виде:
	\[
	\begin{pmatrix}
		1	& x_0	& x_0^2	& ...	& x_0^n \\
		1	& x_1	& x_1^2	& ...	& x_1^n \\
		...	& ...	& ...	&\ddots	& ...	\\
		1	& x_n	& x_n^2	& ...	& x_n^n \\
	\end{pmatrix}
	\begin{pmatrix}
		a_0 \\
		a_1 \\
		... \\
		a_n \\
	\end{pmatrix}
	=
	\begin{pmatrix}
		f(x_0)	\\
		f(x_1)	\\
		...	\\
		f(x_n)	\\
	\end{pmatrix}
	\]

	Определитель этой матрицы -- определитель Вандермонда -- не равен нулю,
	так как $x_i\ne x_j$ при $i\ne j$. Это необходимо и достаточно для
	существования и единственности решения.\qed

\subsection{Интерполяционный многочлен Лагранжа}
В теореме \eqref{eq:polynominal_theorem}, мы доказали, что решение ЗАИ
существует, но там не было ни слова о том, как его искать. Считать методом
Крамера -- слишком долго, чтобы не наскучило. Один из способ определить
интерполяционный полином предложил Луи Лагранж.\newline

\begin{theorem}[о представлении в форме Лагранжа]
	При $n=m$ решение ЗАИ представимо в \textbf{форме Лагранжа}:
	\[P_n(x)=\sum_{k=0}^{n}\frac{\omega(x)}{(x-x_k)\omega'(x_k)}f(x_k),\]
	где
	\[\omega(x)=\prod_{l=0}^{n}(x-x_l)\]
	-- полином степени $n+1$.
\end{theorem}

\beginproof

	Для каждого $k\in\{0,1,...,n\}$ рассмотрим частный случай ЗАИ:
	\[P_{n,k}(x_i)=\delta_{k,i},\;i\in\{0,1,...,n\},\]
	где $\delta_{k,i}$ -- символ Кронекера. Так как полином $P_{n,k}$
	степени $n$ по условию имеет $n$ корней $\{x_0, x_1, ..., x_{k-1},
	x_{k+1}, ..., x_n\}$, он может быть представлен в виде произведения
	мономов:
	\[P_{n,k}(x)=q_k\prod_{\substack{l=0\\ l\neq k}}^{n}(x-x_l),\]
	где $q_k$ берётся из условия $P_{n,k}(x_k)=1$:
	\[q_k=\frac{1}{\prod_{\substack{l=0\\ l\neq k}}^{n}(x-x_l)}.\]
	Обозначив $\omega(x)=\prod_{l=0}^{n}(x-x_l)$, перепишем многочлен:
	\[P_{n,k}(x)=\frac{\omega(x)}{(x-x_k)\omega'(x_k)}f(x_k).\]
	Очевидно, что линейная комбинация
	\[P_n(x)=\sum_{k=0}^{n}P_{n,k}(x)f(x_k)\]
	полиномов -- многочлен степени $n$. А так как $\forall i\in\{0,...,n\}$
	\[P_n(x_i)=\sum_{k=0}^{n}P_{n,k}(x_i)f(x_k)=\sum_{k=0}^{n}\delta_{i,k}
	f(x_k)=f(x_i),\]
	полином является интерполянтом функции $f(x)$.\qed

\end{document}
