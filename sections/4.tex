\documentclass[../main.tex]{subfile}

\begin{document}

\section{Методы численного дифференцирования}
Как и интегрирование, дифференцирование функции тоже далеко не всегда можно
провести аналитическими методами, например, если функция задана таблично. Также
формулы численного дифференцирования используются для решения дифференциальных
уравнений.

\subsection{Разностные операторы}
\begin{define}
	\textbf{Разностный оператор с $k$-ым порядком аппроксимации} -- оператор
	вида $\Lambda_h[y(x)]$, аппроксимирующий дифференциальный оператор вида
	$F[y(x)]$ так, что для всех достаточно гладких функций $y(x)$
	выполняется условие
	\[\big|\Lambda_h[y(x)]-F[y(x)]\big|\le O(h^k),\]
	и при этом существуют такие $y(x)$, что в этом неравенстве достигается
	равенство.
\end{define}

\end{document}
