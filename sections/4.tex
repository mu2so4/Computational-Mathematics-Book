\documentclass[../main.tex]{subfile}

\begin{document}

\section{Методы численного дифференцирования}
Как и интегрирование, дифференцирование функции тоже далеко не всегда можно
провести аналитическими методами, например, если функция задана таблично. Также
формулы численного дифференцирования используются для решения дифференциальных
уравнений.

\subsection{Разностные операторы}
\begin{define}
	\textbf{Разностный оператор} $\Lambda_h[y(x)]$ с $k$-ым порядком
	аппроксимации \textbf{аппроксимирует} дифференциальный оператор
	$F[y(x)]$, если для всех достаточно гладких функций $y(x)$ выполняется
	условие
	\[\big|\Lambda_h[y(x)]-F[y(x)]\big|\le O(h^k),\]
	и при этом для некоторых $y(x)$ достигается равенство.
\end{define}

\begin{define}
	Разность $\Lambda_h[y(x)]-F[y(x)]$ называется \textbf{погрешностью}
	разностного оператора.
\end{define}

\begin{define}
	\textbf{Главный член ошибки} или \textbf{невязки} -- первый член
	погрешности разностного оператора, который содержит в себе производную
	$y(x)$ наименьшего порядка. Обозначается как $\Delta_h[y(x)]$.

	Если $\Lambda_h[y(x)]$ аппроксимирует $n$-ую производную функции
	$y(x)\in C^{n+k}$ с $k$-ым порядком аппроксимации, то $\Delta_h[y(x)]=
	c(h)y^{(n+k)}(x)$, где $c(h)$ -- константа, зависящая от $h$.
\end{define}

\begin{define} \label{eq:simpliest_difference_operators}
	Определим простейщие разностные операторы 1-го порядка:
	\[\Lambda_h^+[y(x)]=\frac{y(x+h)-y(x)}{h},\qquad
	\Lambda_h^-[y(x)]=\frac{y(x)-y(x-h)}{h}.\]
\end{define}

\begin{lemma}
	Данные разностные операторы аппроксимируют первую производную для всяких
	$y(x)\in C^3$ с первым порядком аппроксимации.
\end{lemma}

\begin{proof}
	Для определённости рассмотрим $\Lambda_h^+[y(x)]$; для \\
	$\Lambda_h^-[y(x)]$ доказательство аналогичное. Разложим функцию
	$y(x+h)$ в ряд Тейлора по $h$ при $h=0$ с точностью до $h^3$:
	\[f(x+h)=y(x)+hy'(x)+\frac{h^2}{2}y''(x)+\frac{h^3}{6}y'''(x)
	+O(h^4).\]
	Тогда
	\[\Lambda_h^+[y(x)]=y'(x)+\frac{h}{2}y''(x)+\frac{h^2}{6}
	y'''(x)+O(h^3),\]
	\[\Lambda_h[y(x)]-F[y(x)]=\frac{h}{2}y''(x)+\frac
	{h^2}{6}y'''(x)+O(h^3)\Rightarrow \Delta_h[y(x)]=\frac{h}{2}y''(x).\]
\end{proof}

\begin{define}
	Разностные операторы из \eqref{eq:simpliest_difference_operators}
	относятся к \textbf{операторам направленной разности} (вперёд и назад
	соответственно). Из них можно составить оператор \textbf{симметричной}
	разности:
	\[\Lambda_h^*[y(x)]=\frac{\Lambda_h^+[y(x)]+\Lambda_h^-[y(x)]}{2}=
	\frac{y(x+h)-y(x-h)}{2h}.\]
\end{define}

\begin{lemma}
	Данный разностный оператор аппроксимирует первую производную для всяких
	$y(x)\in C^5$ со вторым порядком аппроксимации.
\end{lemma}

\begin{proof}
	Воспользовавшись разложением
	\[y(x\pm h)=y(x)\pm hy'(x)+\frac{h^2}{2!}y''(x)\pm\frac{h^3}{3!}y'''(x)+
	\frac{h^4}{4!}y^{(4)}(x)\pm\frac{h^5}{5!}y^{(5)}(x)+O(h^6),\]
	получим, что
	\[\Lambda_h^*[y(x)]=y'(x)+\frac{h^2}{6}y'''(x)+\frac{h^4}{120}y^{(5)}(x)
	+O(h^5).\]
\end{proof}

\begin{remark}
	Повышение порядка аппроксимации требует повышение гладкости
	аппроксимируемой функции.
\end{remark}

\end{document}
