\documentclass[../main.tex]{subfile}

\begin{document}

\section{Методы численного дифференцирования}
Как и интегрирование, дифференцирование функции тоже далеко не всегда можно
провести аналитическими методами, например, если функция задана таблично. Также
формулы численного дифференцирования используются для решения дифференциальных
уравнений.

\subsection{Разностные операторы}
\begin{define}
	\textbf{Разностный оператор} $\Lambda_h[y(x)]$ с $k$-ым порядком
	аппроксимации \textbf{аппроксимирует} дифференциальный оператор
	$F[y(x)]$, если для всех достаточно гладких функций $y(x)$ выполняется
	условие
	\[\big|\Lambda_h[y(x)]-F[y(x)]\big|\le O(h^k),\]
	и при этом для некоторых $y(x)$ достигается равенство.
\end{define}

\begin{define}
	Разность $\Lambda_h[y(x)]-F[y(x)]$ называется \textbf{погрешностью}
	разностного оператора.
\end{define}

\begin{define}
	\textbf{Главный член ошибки} или \textbf{невязки} -- первый член
	погрешности разностного оператора, который содержит в себе производную
	$y(x)$ наименьшего порядка. Обозначается как $\Delta_h[y(x)]$.

	Если $\Lambda_h[y(x)]$ аппроксимирует $n$-ую производную функции
	$y(x)\in C^{n+k}$ с $k$-ым порядком аппроксимации, то $\Delta_h[y(x)]=
	c(h)y^{(n+k)}(x)$, где $c(h)$ -- константа, зависящая от $h$.
\end{define}
\newpage

Рассмотрим простейшие аппроксимации дифференциальных операторов первого порядка.

\begin{define}
	\textbf{Направленный вперёд разностный оператор 1-го порядка} --
	оператор вида
	\[\Lambda_h^+[y(x)]=\frac{y(x+h)-y(x)}{h}.\]

	\textbf{Направленный назад разностный оператор 1-го порядка} --
	оператор вида
	\[\Lambda_h^-[y(x)]=\frac{y(x-h)-y(x)}{h}.\]
\end{define}

\end{document}
