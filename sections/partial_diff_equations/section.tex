\documentclass{article}

\begin{document}
\section{Численное решение дифференциальных уравнений в частных производных}

Уравнения в частных производных бывают эллиптическими и гиперболическими. Мы
рассмотрим только гиперболические, начнём с самого простого из них.

\subsection{Линейное уравнение переноса}
\begin{define}\label{eq:convection-diffusion_equation}
	\textbf{Линейное уравнение переноса} и задача Коши к нему имеет вид
	\[u_t+au_x=0,\quad t>0,\quad a=const>0,\quad u(x,0)=v(x)\in
	\mathbb C^1.\]
\end{define}

\begin{lemma}\label{eq:CDE_solution}
	Задача Коши к линейному уравнению переноса
	\eqref{eq:convection-diffusion_equation} имеет решение
	\[\boxed{u(x,t)=v(x-at)}.\]
\end{lemma}

\begin{proof}
	Найдём характеристику данного уравнения, то есть производную
	координаты по времени:
	\[\frac{dx}{dt}=a\Rightarrow x=at+x_0,\]
	где $x_0=const$, то есть характеристика -- это семейство прямых с
	заданным углом к положительному направлению оси $x$.

	Продифференцируем $u$ по $t$ полностью, опираясь на нашу характеристику:
	\[\frac{d}{dt}u(at+x_0,t)=u_t+au_x=0.\]

	Тогда получается, что $u=const$ на каждой отдельно взятой
	характеристике, то есть, для любой точки $(x,t)$ существует точка
	$(x_0,0)$ такая, что $u(x,t)=u(x_0,0)=v(x_0)$. Тогда $x_0=x-at$ и
	$u(x,t)=v(x-at)$.
\end{proof}

Почему линейное уравнение переноса имеет такое название? Потому что на графике
это выглядит так, будто мы осуществили параллельный перенос графика, причём
перенос с постоянной скоростью во всех точках: \\

\subfile{graph-CDE}

\subsection{Линейные гиперболические системы}
\begin{define}
	Скаляр $\lambda$ и ненулевой вектор $v$ называются собственным
	значением и собственным вектором матрицы $A$ соответственно, если для
	них выполняется равенство
	\[A\boldsymbol v=\lambda \boldsymbol v.\]
\end{define}

\begin{define}\label{eq:hyperbolic_system}
	Линейная система \[\boldsymbol u_t+A\boldsymbol u_x=0,\]
	где $\boldsymbol u\in \mathbb R^n$, называется
	\textbf{гиперболической}, если все собственные значения матрицы $A$
	действительные, а также существует базис пространства $\mathbb R^n$
	из её собственных векторов.
\end{define}

\begin{define}\label{eq:strict_hyperbolic_system}
	Гиперболическая система \eqref{eq:hyperbolic_system} является
	\textbf{строгой}, если все собственные значения матрицы $A$ различные.
\end{define}

В таком случае принято нумеровать собственные значения по порядку их
возрастания, то есть $\lambda_1<\lambda_2<...<\lambda_n$.

\begin{lemma}\label{eq:hyperbolic_invariant_form}
	Строгую линейную гиперболическую систему
	\eqref{eq:strict_hyperbolic_system} можно записать в виде
	\[\boxed{\boldsymbol\omega_t+\Lambda\boldsymbol\omega_x=0},\]
	где $\boldsymbol\omega=S\boldsymbol u$ -- вектор инвариантов, $S$ --
	матрица, чьи строки составлены из собственных векторов матрицы $A$,
	а $\Lambda$ -- диагональная матрица соответствующих им собственных
	значений.
\end{lemma}

\begin{proof}
	Пусть $\boldsymbol l^1,...,\boldsymbol l^n$ -- собственные векторы
	матрицы $A$, а $\lambda_1,...\lambda_n$ -- соответствующие им
	собственные значения. Пусть $S$ -- это матрица, чьи строки составлены из
	собственных векторов матрицы $A$, то есть $S=(\boldsymbol l^1,..,
	\boldsymbol l^n)^T$, а элементы диагональной матрицы $\Lambda$
	составлены из соответствующих собственных значений. Тогда очевидно, что
	\[SA=\Lambda S.\]

	Обозначим $\boldsymbol\omega=S\boldsymbol u=(\omega^1,...,\omega^n)^T$
	-- вектор инвариантов, где $\omega_i$ -- инвариант относительно $i$
	характеристики. Матрица $S$ невырожденная, так как собственные векторы
	линейно независимые, так что на неё можно домножить слева
	гиперболическую систему и немного её преобразовать:

	\[S\boldsymbol u_t+SA\boldsymbol u_x=S\boldsymbol u_t+A\Lambda
	\boldsymbol u_x=\boldsymbol\omega_t+\Lambda\boldsymbol\omega_x,\]
	и мы получили требуемую запись уравнения.
\end{proof}

\begin{theorem}[решения задачи Коши строгой линейной гиперболической системы]
	Решение системы \eqref{eq:strict_hyperbolic_system} с начальным условием
	$\boldsymbol u(x,0)=\boldsymbol{v}(x)$ имеет вид
	\[\boldsymbol u(x,t)=S^{-1}\boldsymbol\omega,\]
	где $S$ -- матрица, чьи строки составлены из собственных векторов
	матрицы $A$, $\boldsymbol\omega^i(x,t)=p_i(x-\lambda_i t)$,\quad $p(x)=
	\boldsymbol\omega(x,0)=S\boldsymbol v$, а $\lambda_i$ --
	соответствующие собственные значения этой матрицы.
\end{theorem}

\begin{proof}
	После преобразования строгой гиперболической системы, проведённого в
	лемме \eqref{eq:hyperbolic_invariant_form}, полученная квазилинейная
	гиперболическая система распадается на $n$ независимых друг от друга
	линейных уравнений переноса относительно инвариантов вдоль своих
	характеристик. Каждое уравнение имеет вид
	\[(\omega^i)_t+\lambda_i(\omega^i)_x=0,\]
	то есть является линейным уравнением переноса. Начальное значение $v(x)$
	преобразовывается в
	\[\boldsymbol p(x)=\boldsymbol\omega(x,0)=S\boldsymbol v.\]

	По лемме \eqref{eq:CDE_solution}, решение такого уравнения переноса
	с данным начальным условием равно
	\[\omega^i(x,t)=p_i(x-\lambda_i t),\]
	и теперь, чтобы от $\boldsymbol\omega$ обратно перейти к
	$\boldsymbol u$, нужно домножить на $S^{-1}$:
	\[\boldsymbol u=S^{-1}\boldsymbol\omega.\]
\end{proof}

Графически решение гиперболического решения выглядит так: \\

\subfile{graph-hyperbolic_system_solution} \\

Читается этот график следующим образом. Точка $(x,t)$ является нашим начальным
условием. В неё необходимо привести все характеристики, которые
''распростраяняются'' по прямым. $\lambda_i$ становятся просто тангенсами
углов наклона этих самых прямых. При таких значениях мы получаем $x_1,...,x_n$
в качестве решения задачи Коши.

\begin{example}\label{eq:wave_equation_example}
	В качестве примера рассмотрим волновое уравнение
	\[u_{tt}=a^2u_{xx},\quad a=const\ne 0,\]
	которое нам известно с курса физики 4 семестра и эквивалентно
	системе
	\[\begin{cases}
		u_t+v_x=0, \\
		v_t+a^2u_x=0. \\
	\end{cases}\]

	Здесь
	\[\boldsymbol{u}=
		\begin{pmatrix}
			u \\
			v \\
		\end{pmatrix},
	\qquad
	A=
		\begin{pmatrix}
			0 & 1 \\
			a^2 & 0 \\
		\end{pmatrix}.
	\]

	Собственные значения и собственные вектора матрицы $A$ равны
	\[\lambda_1=-a,\quad \lambda_2=a,\qquad
	\boldsymbol{l}_1=(a,-1),\quad \boldsymbol{l}_2=(a,1),\]
	\[
		S=
		\begin{pmatrix}
			a & -1 \\
			a & 1  \\
		\end{pmatrix}.
	\]

	$|S|\ne 0$ при $a\ne 0$, так что исходная система гиперболическая.

	Вектор инвариантов $\boldsymbol{\omega}=S\boldsymbol{u}$ имеет
	компоненты
	\[\omega^1=au-v,\quad \omega^2=au+v,\]

	первая из которых сохраняется вдоль семейства характеристик
	$\lambda_1=-a\Rightarrow x+at=C$, а вторая -- вдоль семейства
	характеристик $\lambda_2=a\Rightarrow x-at=C$. \\

	\subfile{graph-wave_equation}
	
	В результате получаем, что
	\[\omega^1(x_0,t_0)=\omega^1(x_1,0)=p_1(x+at),\qquad
	  \omega^2(x_0,t_0)=\omega^2(x_2,0)=p_2(x-at),\]
	где $p_1,\;p_2\in \mathbb C^2$ -- функции, задающие начальные значения
	инвариантов.
	
	Общее решение исходной схемы, таким образом, равно
	\[u(x,t)=\frac{1}{2a}\big(\omega^2(x,t)+\omega^1(x,t)\big)=
	\frac{1}{2a}\big(p_2(x-at)+p_1(x+at)\big),\]
	\[v(x,t)=\frac{1}{2}\big(\omega^2(x,t)-\omega^1(x,t)\big)=
	\frac{1}{2}\big(p_2(x-at)-p_1(x+at)\big).\]
	
	Если же для этой системы поставить задачу Коши с начальными данными
	\[u(x,0)=u_0(x),\quad v(x,0)=v_0(x),\quad u,v\in\mathbb C^2,\]
	то её решение имеет вид
	\[u(x,t)=\frac{1}{2}\big(u_0(x-at)+u_0(x+at)\big)+
	\frac{1}{2a}\big(v_0(x-at)-v_0(x+at)\big),\]
	\[v(x,t)=\frac{1}{2}\Big(v_0(x-at)+v_0(x+at)+
	a\big(u_0(x-at)-u_0(x+at)\big)\Big).\]
\end{example}

\subsection{Начально-краевая задача для линейной гиперболической системы ДУ}

\begin{define}
	\textbf{Начально-краевая задача} -- задача о нахождении решения
	дифференциального уравнения такого, что оно удовлетворяет начальным
	и граничным условиям.
\end{define}

\begin{theorem}\label{eq:boundary_problem_condition}
	Пусть матрица $A$ из уравнения \eqref{eq:hyperbolic_system} имеет $n$
	различных собственных значений, из которых $k$ отрицательные, а $m$
	неположительные. Пронумеруем их в порядке возрастания:
	\[\lambda_1\le...\le\lambda_k<0=\lambda_{k+1}=...=\lambda_m<
	\lambda_{m+1}\le...\le\lambda_n\].

	Обозначим соответствующие собственные векторы как $\boldsymbol l_1,...,
	\boldsymbol l_n$,
	а компоненты вектора инвариантов из $\boldsymbol\omega=S\boldsymbol u$
	как $\omega_1,...,\omega_n$.

	Чтобы корректно задать краевую задачу, необходимо и достаточно, чтобы

	\begin{enumerate}[nosep]
		\item
			Количество граничных условий на левой границе совпадает с
			количеством положительных характеристик, а на правой --
			с количеством отрицательных, то есть
			\[
				\begin{cases}
					\big(\boldsymbol\varphi^i,
						\boldsymbol u(0,t)\big)=f_i(t),&
						i\in\overline{1,k}, \\
					\big(\boldsymbol\varphi^i,
						\boldsymbol u(X,t)\big)=f_i(t),&
						i\in\overline{m+1,n}, \\
				\end{cases}
			\]
			где $f_i(t)$ -- некоторые скалярные функции,
			$\boldsymbol\varphi^i$ -- некоторые константные векторы,
			а $(\boldsymbol a,\boldsymbol b)$ -- скалярное
			произведение векторов,
		\item
			На левой границе $x=0$ должно быть верно
			\[|\boldsymbol l^1,...,\boldsymbol l^m,
			\boldsymbol\varphi^{m+1},...,
			\boldsymbol\varphi^n|\ne 0,\]
			где $|\boldsymbol a_1,...,\boldsymbol a_k|$ --
			определитель матриц, чьи столбцы составлены из
			$k$-мерных векторов $\boldsymbol a_1,...,
			\boldsymbol a_k$,
		\item
			На правой границе $x=X$ должно быть верно
			\[|\boldsymbol\varphi^1,...,\boldsymbol\varphi^k,
			\boldsymbol l^{k+1},...,\boldsymbol l^n|\ne 0.\]
	\end{enumerate}
\end{theorem}

\begin{proof}
	На левую границу $x=0$ изнутри области приходит первые $k$
	характеристик. Нулевые характеристики можно провести к любой границе,
	то есть для них можно не задавать граничные условия. Остальные $n-m$
	условий мы определим в граничных условиях. Аналогично сделаем и с правой
	границей, там понадобится уже $k$ граничных условий.

	Также корректность краевой задачи означает, что вектор $\boldsymbol u$
	однозначно определяется на границах. Необходимо, чтобы система уравнений
	была разрешима, что достижимо при ненулевом определителе матрицы.

	На левой границе будет
	\[
	\begin{cases}
		\big(\boldsymbol l^i,\boldsymbol u(0,t)\big)=\omega_i, &
			i\in\overline{1,m}, \\
		\big(\boldsymbol\varphi^i,\boldsymbol u(0,t)\big)=f_i(t), &
			i\in\overline{m+1,n}. \\
	\end{cases}
	\]

	На правой же границе будет
	\[
	\begin{cases}
		\big(\boldsymbol\varphi^i,\boldsymbol u(X,t)\big)=f_i(t), &
			i\in\overline{1,k}, \\
		\big(\boldsymbol l^i\boldsymbol u(X,t)\big)=\omega_i, &
			i\in\overline{k+1,n}, \\
	\end{cases}
	\]

	Чтобы системы были разрешимы, нужно, чтобы определители матриц были
	отличны от нуля, благодаря чему и получаем необходимые условия.
\end{proof}

\begin{example}
	Рассмотрим начально-краевую задачу на границе $[0,X]$ для линейного
	уравнения переноса \eqref{eq:convection-diffusion_equation}
	\[u_t+au_x=0.\]
	Характеристика у нас одна, и она положительна. Изобразим её возможные
	положения: \\

	\subfile{graph-CDE_boundary_problem} \\

	Очевидно, что для корректной постановки начально-краевой задачи
	начальное условие необходимо пускать от правой границы $X$, то есть
	\[\varphi u(X,t)=f(t).\]
\end{example}

\begin{example}\label{eq:boundary_problem_example}
	Рассмотрим начально-краевую задачу на $[0,X]$ для волнового уравнения
	(уравнения акустики), которое уже было в примере
	\eqref{eq:wave_equation_example}:
	\[\begin{cases}
		u_t+v_x=0, \\
		v_t+a^2u_x=0. \\
	\end{cases}\]

	$a^2$ положим равным единице, тогда
	\[\boldsymbol{u}=
		\begin{pmatrix}
			u \\
			v \\
		\end{pmatrix},
	\qquad
	A=
		\begin{pmatrix}
			0 & 1 \\
			1 & 0 \\
		\end{pmatrix}.
	\]

	Собственные значения и собственные векторы матрицы $A$ равны
	\[\lambda_1=-1,\quad \lambda_2=1,\qquad
	\boldsymbol{l}_1=(1,-1),\quad \boldsymbol{l}_2=(1,1).\]

	На каждой границе будет по одному граничному условию. Для корректной
	поставновки граничных условий нужен вектор инвариантов
	$\boldsymbol\omega$. Его компоненты равны
	\[\omega_1=u-v,\qquad \omega_2=u+v.\]

	На левой границе $(\boldsymbol\varphi,\boldsymbol u)=f_1$, тогда
	\[
		\begin{cases}
			\varphi_1 u+\varphi_2 v=f_1, \\
			u-v=\omega_1. \\
		\end{cases}
	\]

	Применим условие корректности:
	\[
		\begin{vmatrix}
			\varphi_1 & \varphi_2 \\
			1	& -1	\\
		\end{vmatrix}
		\ne 0\quad\Leftrightarrow\quad
		\varphi_1\ne -\varphi_2.
	\]

	Аналогично на правой границе при $(\boldsymbol\varphi,\boldsymbol u)=
	f_2$
	\[
		\begin{cases}
			u+v=\omega_2, \\
			\varphi_1 u+\varphi_2 v=f_2. \\
		\end{cases}
		\Rightarrow\quad
		\begin{vmatrix}
			1	& 1	\\
			\varphi_1 & \varphi_2 \\
		\end{vmatrix}
		\ne 0\quad\Leftrightarrow\quad
		\varphi_1\ne \varphi_2.
	\]
\end{example}

\subsection{Разностные схемы, аппроксимирующие уравнения переноса}
Это любимая тема В. Остапенко. Рассматривать построение разностных схем
мы будем лишь для линейного уравнения переноса
\eqref{eq:convection-diffusion_equation}, однако правильность его решения лежит
в основе более сложных задач.

Несмотря на то что мы аппроксимируем функцию, вполне естественно требовать от
аппроксимации сохранения свойств точной функции, например, монотонности.

Здесь и далее будет использоваться обозначение $u_j^n=u(jh,n\tau)$, где $h$ --
шаг по координате, а $\tau$ -- шаг по времени. При этом значения $u^0_j$ при
некоторых $j$ являются начальными условиями.

\begin{define}\label{eq:two_layer_scheme}
	\textbf{Двуслойная по времени} разностная схема -- разностная схема вида
	\[u_j^{n+1}=\sum_{m\in M}C_mu_{j+m}^n.\]
\end{define}

\begin{define}\label{eq:monotonic_scheme}
	Разностная схема \eqref{eq:two_layer_scheme} называется
	\textbf{монотонной}, если она при любом $j$ переводит монотонную функцию
	$u_j^n$ на следующий временной слой как функцию $u_j^{n+1}$, которая
	монотонна с тем же знаком (монотонно возрастающая функция не переходит в
	монотонно убывающую, и наоборот).
\end{define}

\begin{theorem}[критерий монотонности разностной схемы]
\label{eq:monotonic_scheme_criterion}
	Чтобы разностная схема \eqref{eq:two_layer_scheme} была монотонной,
	необходимо и достаточно, чтобы все коэффициенты $C_j$ были
	неотрицательные.
\end{theorem}

\begin{proof}
	Для определённости допустим, что $u_j^n$ -- монотонно возрастающая
	функция. Для монотонно убывающих функций доказательство аналогичное.
	
	\textbf{Необходимость}. Пусть все коэффициенты $C_j$ неотрицательные.
	Обозначим
	\[\Delta\circ u_j=u_j-u_{j-1}.\]
	Если схема монотонно возрастающая, то $u_{j+1}-u_j\ge 0$. Подействуем
	этим оператором на разностную схему. Благодаря его линейности,
	\[\Delta\circ u_j^{n+1}=\sum_{m\in M}C_m\Delta u_{j+m}^{n}\ge 0
	\Rightarrow u_j^{n+1}\text{ монотонно возрастающая}.\]

	\textbf{Достаточность}. Докажем от противного. Пусть $C_k<0$, а функция
	$u_j^n$ монотонно возрастающая. Здесь дана разность, направленная
	вперёд. Зададим $u_j^n$ так, что
	\[
		u_j^n=
		\begin{cases}
			0, & j < 0, \\
			1, & j\ge 0. \\
		\end{cases}
	\]
	Очевидно, что на данном временном слое функция монотонная и что
	\[\exists m\in M:j+m=0\Rightarrow
	\Delta u_j^n=C_{-j}\underset{1}{\underbrace{\Delta u_0^n}}.\]

	В том случае, если мы выберем $-j=k$, то получим, что
	$\Delta u_{-k}^n<0$, что является нарушением условия монотонности
	разностной схемы. Пришли к противоречию.
\end{proof}

\begin{define}\label{eq:CDE_forward_scheme}
	Простейщая разностная схема по потоку, аппроксимирующая линейное
	уравнение переноса, имеет вид
	\[\frac{u_j^{n+1}-u_j^n}{\tau}+a\frac{u_{j+1}^n-u_j^n}{h}=0.\]

	Шаблон схемы имеет вид \\

	\subfile{graph-forward_scheme}
\end{define}

\begin{define}\label{eq:godunov_scheme}
	Простейщая разностная схема против потока, аппроксимирующая линейное
	уравнение переноса, называется \textbf{схемой Годунова} и имеет вид
	\[\frac{u_j^{n+1}-u_j^n}{\tau}+a\frac{u_j^n-u_{j-1}^n}{h}=0.\]

	Шаблон схемы имеет вид \\

	\subfile{graph-godunov_scheme}
\end{define}


\begin{define}\label{eq:courant_number}
	\textbf{Числом Куранта} в разностной схеме линейного уравнения переноса
	называется число $\mathlarger{r=\frac{a\tau}{h}}$.
\end{define}

\begin{lemma}
	Разностная схема по потоку \eqref{eq:CDE_forward_scheme} не монотонна.
\end{lemma}

\begin{proof}\label{eq:CDE_forward_scheme_lemma}
	Запишем схему в явном виде:
	\[u_j^{n+1}=u_j^n-r(u_{j+1}^n-u_j^n).\]

	Найдём аппроксимацию сдвига функции
	\[v(x)=
		\begin{cases}
			0, & x<0, \\
			1, & x\ge 0. \\
		\end{cases}
	\]

	Схематично точное решение выглядит так, будто мы передвинули
	''ступеньку'':

	\subfile{graph-unit_step_function}

	Проверим, сохраняет ли разностная схема монотонность. Зададим нижний
	временной слой точно:
	\[u_j^0(x)=
		\begin{cases}
			0, & j<0, \\
			1, & j\ge 0. \\
		\end{cases}
	\]

	В отличие от $v(x)$, эта функция не непрерывная, а дискретная.

	Теперь зададим $u_j^1$, которая на следующем временном слое:
	\[u_j^1=u_j^0-r(u_{j+1}^0-u_j^0).\]

	Если посчитать все значения первого временного слоя, то получим, что
	\[u_j^1(x)=
		\begin{cases}
			0, & j<-1, \\
			-r,& j=-1, \\
			1, & j>-1, \\
		\end{cases}
	\]
	что сразу говорит о немонотонности первого временного слоя, так как
	$r>0$. Значит, разностная схема не монотонна.

	По критерию монотонности, мы имеем отрицательный коэффициент при
	$u_{j+1}^n$, что также говорит о немонотонности схемы.
\end{proof}

\begin{theorem}[условие Куранта]\label{eq:courant_condition}
	Если число Куранта $r\le 1$, то схема Годунова
	\eqref{eq:godunov_scheme} монотонная.
\end{theorem}

\begin{proof}
	В явном виде схема имеет вид
	\[u_j^{n+1}=u_j^n-r(u_j^n-u_{j-1}^n).\]

	Проделаем те же шаги, что и в лемме \eqref{eq:CDE_forward_scheme_lemma},
	и получим, что на первом временном слое
	\[u_j^1(x)=
		\begin{cases}
			0,   & j<0, \\
			1-r, & j=0, \\
			1,   & j>0. \\
		\end{cases}
	\]

	Отсюда вытекает условие, при котором функция на данном временном слое
	останется монотонной: $r\le 1$. То же условие на число Куранта можно
	получить и из критерия монотонной разностной схемы.
\end{proof}

\subsubsection{Схема Лакса}

Рассмотрим другую схему, аппроксимирующую линейное уравнение переноса:
\[\frac{u_j^{n+1}-u_j^n}{\tau}+a\frac{u_{j+1}^n-u_{j-1}^n}{2h}=0.\]
Очевидно, что она немонотонная при любом числе Куранта. Были приняты несколько
попыток монотонизировать данную схему. Первый вариант предложил Питер Лакс в
1953 году.

\begin{define}\label{eq:lax_scheme}
	\textbf{Схемой Лакса} называется разностная схема, аппроксимирующее
	линейное уравнение переноса, вида
	\[\frac{u_j^{n+1}-\frac{u_{j-1}^n+u_{j+1}^n}{2}}{\tau}+
	a\frac{u_{j+1}^n-u_{j-1}^n}{2h}=0.\]
\end{define}

\begin{lemma}
	Схема Лакса монотонна, если число Куранта $r\le 1$.
\end{lemma}

\begin{proof}
	Запишем схем в операторном виде:
	\[u_j^{n+1}=\frac{u_{j-1}^n+u_{j+1}^n}{2}-
	\frac{r}{2}(u_{j+1}^n-u_{j-1}^n)=
	\frac{1}{2}\big((1+r)u_{j-1}^n+(1-r)u_{j+1}^n\big).\]

	Коэффициент $1+r$ всегда положителен, а вот $1-r$ по критерию
	монотонности \eqref{eq:monotonic_scheme_criterion} неотрицателен при
	$r\le 1$.
\end{proof}

Сейчас мы рассмотрим, чем связаны схема Годунова \eqref{eq:godunov_scheme} со
схемой Лакса. Для этого необходимо ввести ещё одну разностную схему, немного
испорченную.

\begin{define}\label{eq:god_lax_scheme}
	\textbf{Схема с искусственной вязкостью} -- разностная схема,
	аппроксимирующее линейное уравнение переноса, вида
	\[\frac{u_j^{n+1}-u_j^n}{\tau}+a\frac{u_j^n-u_{j-1}^n}{h}=
	Ca\frac{u_{j+1}^n-2u_j^n+u_{j-1}^n}{h}.\]
\end{define}

После равенства стоит аппроксимация второй производной, и она не нарушает
аппроксимацию, разве что аппроксимация по пространству теперь первого порядка,
что дало дополнительное $h$ перед дробью. Поэтому $h$ в дроби без квадрата.

\begin{lemma}
	Чтобы Схема с искусственной вязкостью была монотонной, необходимо
	и достаточно, чтобы число Куранта $r<1$, а
	$C\in[\frac{1}{2},\frac{1}{2r}]$.
\end{lemma}

\begin{proof}
	Запишем схему в операторном виде:
	\[u_j^{n+1}=u_j^n-\frac{r}{2}(u_{j+1}^n-u_{j-1}^n)+Cr(u_{j+1}^n-2u_j^n+
	u_{j-1}^n)=\]
	\[=r\Big(C+\frac{1}{2}\Big)u_{j-1}^n+\Big(1-2Cr\Big)u_j^n+r\Big(C-
	\frac{1}{2}\Big)u_{j+1}^n.\]

	Теперь условия на коэффициенты благодаря критерию монотонности
	\eqref{eq:monotonic_scheme_criterion} становятся очевидными.
	Отметим только, что при $r=1$ схема становится точной, а при таком
	значении разностные схемы не считают.
\end{proof}

Теперь, если мы в этой схеме возьмём $C=\frac{1}{2}$, то получится схема
Годунова, а если $C=\frac{1}{2r}$, то выйдет схема Лакса. Годунов назвал свою
схему ''наилучшей'', так как она имеет меньшую искусственную вязкость.

В чём различие этих схем? Покажем это на функции
\[v(x)=
	\begin{cases}
		1, & x\le 0, \\
		0, & x>0. \\
	\end{cases}
\]

Сравним точное решение с работой обеих схем, взяв подходящий временной слой:

\subfile{graph-godunov_vs_lax}

Всякая схема ''размазывает'' функцию, и чем больше временных слоёв было
преодолено, тем больше ширина размазывания. Профиль решения схемой Годунова
где-то всё же повторяет точное решение. Схема Лакса же меняет значение через
раз, и это не очень хорошо.

Почему это не очень хорошо? Взглянем на разностные производные всех трёх
функций. Разностная производная вычисляется по формуле
\[\omega_j^n=\frac{\Delta u_j^n}{h}=\frac{u_{j+1}^n-u_j^n}{h}.\]

Если у точной схемы разностная производная -- это просто скачок на нуле, у
схемы Годунова -- сначала нули, а потом по убыванию по модулю ненулевые
значения, то вот у схемы Лакса после нул нулевые значения производных
чередуются с ненулевыми, что очень похоже на ёжика. И такая производная не есть
хорошо.

\begin{define}\label{eq:strongly_monotonic_scheme}
	\textbf{Сильно монотонная разностная схема} $n$-го порядка -- монотонная
	разностная схема \eqref{eq:monotonic_scheme}, чьё каждое численное
	решение на любом временном слое переводит на следующий временной слой
	без увеличения количества локальных экстремумов, если их не более $n$.
\end{define}

Очевидно, что схема Годунова сильно монотонна, а вот схема Лакса -- нет. Чтобы
это окончательно подтвердить или опровергнуть, необходим критерий.

\begin{theorem}[критерий сильной монотонности]
	Разностная схема \eqref{eq:monotonic_scheme}, чей шаблон дискретно
	неперывный (не имеет пустот между элементами шаблона), является
	сильно монотонной 1-го тогда и только тогда, когда
	\[\forall m\in M: C_m\ge\sqrt{C_{m-1}C_{m+1}}.\]
\end{theorem}

\noproof

Если мы применим данное условие для схемы с искусственной вязкостью
\eqref{eq:god_lax_scheme}, то получим, что, чтобы схема была сильно монотонной,
необходимо и достаточно того, чтобы
\[1-2Cr\ge\sqrt{C^2-\frac{1}{4}}.\]

Это значит, что внутри отрезка $[\frac{1}{2},\frac{1}{2r}]$ существует пороговая
точка $C'$, до которой все схемы сильно монотонны, а после неё -- нет. Схема
Лакса тогда \eqref{eq:lax_scheme} не является сильно монотонной, а схема
Годунова \eqref{eq:godunov_scheme} -- является.

К сожалению, пока что не удалось вывести критерий сильной монотонности высших
порядков, но Остапенко смог это сделать для трёхточечных разностных схем вида
\[u_j^{n+1}=C_{-1}u_{j-1}^n+C_0u_j^n+C_1u_{j+1}^n,\]
чей критерий сильной монотонности $k$-го порядка выглядит как
\[C_0\ge\alpha_k\sqrt{C_{-1}C_1}.\]
$\{\alpha_k\}$ -- это некоторая последовательность,
причём $\alpha_1=1$, $\alpha_2=\sqrt 2$, а сама последовательность стремится к
2. Предельный случай обозначим отдельно.

\begin{theorem}
	Разностная схема вида
	\[u_j^{n+1}=C_{-1}u_{j-1}^n+C_0u_j^n+C_1u_{j+1}^n.\]
	является абсолютно монотонной по Остапенко, то есть она не увеличивает
	количество локальных экстремумов, сколько бы их ни было, если для
	коэффициентов верно соотношение
	\[C_0\ge 2\sqrt{C_{-1}C_1}.\]
\end{theorem}

\subsubsection{Схема Лакса-Вендроффа}
Это всё были явные схемы первого порядка по времени. Однако есть схемы и более
высоких порядков точности. Одну из них представили Лакс и Вендрофф в 1960 году.

\begin{lemma}[схема Лакса-Вендроффа]\label{eq:lax_wendruff_scheme}
	Схема Лакса-Вендроффа имеет вид
	\[\frac{u_j^{n+1}-u_j^n}{\tau}+a\frac{u_{j+1}^n-u_{j-1}^n}{2h}=
	\frac{a^2\tau}{2}\cdot\frac{u_{j+1}^n-2u_j^n+u_{j-1}}{h^2}.\]
\end{lemma}

\begin{proof}
	Разложим в ряд Тейлора обе производные:
	\[\frac{u(x,t+\tau)-u(x,t)}{\tau}+a\frac{u(x+h,t)-u(x-h,t)}{2h}=
	\cancel{u_t}+\frac{\tau}{2}u_{tt}+\cancel{au_x}+O(\tau^2+h^2).\]

	Нужно избавиться от второго слагаемого, но при этом оставить
	двуслойность схемы. В этом могут помочь дифференциальные следствия,
	которые получаются в ходе дифференцирования линейного уравнения
	переноса. Сначала продифференцируем по пространству:
	\[u_{tx}+au_{xx}=0\quad\Rightarrow\quad au_{tx}=-a^2u_{xx}.\]

	Теперь применим данное равенство в производной уже по времени и
	воспользуемся свойством смешанной производной:
	\[u_{tt}+au_{tx}=0\quad\Rightarrow\quad u_{tt}=a^2u_{xx}.\]

	Осталось провести в разностном операторе данную замену и расписать
	вторую производную по пространству, после чего получится нужная
	запись схемы.
\end{proof}

\begin{lemma}
	Схема Лакса-Вендроффа \eqref{eq:lax_wendruff_scheme} монотонна,
	если число Куранта $r=1$.
\end{lemma}

\begin{proof}
	Запишем схему в операторном виде:
	\[u_j^{n+1}=u_j^n-\frac{r}{2}(u_{j+1}^n-u_{j-1})+
	\frac{r^2}{2}(u_{j+1}^n-2u_j^n+u_{j-1})=\]
	\[=u_{j+1}\frac{r}{2}(r-1)+u_j^n(1-r^2)=u_{j-1}^n\frac{r}{2}(1+r).\]

	Из критерия монотонности \eqref{eq:monotonic_scheme_criterion} получаем
	ограничение на число Куранта.
\end{proof}

\subsubsection{Запрет Годунова}
Сейчас мы получили схему, в которой у $r$ может быть только одно значение, при
котором схема вырождается в точное решение. И это не случайно: Годунов смог
установить важное ограничение на порядок аппроксимации по времени.

\begin{theorem}[Запрет Годунова]
	Среди явных двуслойных по времени линейных разностных схем,
	аппроксимирующих линейное уравнение переноса, при числе Куранта
	$r<1$, не существует монотонных схем с аппроксимацией, большей 1
	(схем повышенной точности).
\end{theorem}

\begin{proof}
	Допустим, что такая схема существует. Пусть разностная схема имеет вид
	\[u_j^{n+1}=\sum_{m\in M}C_m u_{j+m}^n,\quad C_m\ge 0\;\forall m\in M.\]

	Запишем схему в открытой формe. Если схема аппроксимирует со вторым
	порядком, то после разложения в ряд Тейлора
	\[\Lambda_h^\tau[u_h^\tau(x,t)]=O(\tau^2+h^2)=
	\frac{\tau^2}{6}u_{ttt}+\frac{h^2}{6}u_{xxx}+O(...).\]

	Это значит, что если мы для линейного уравнения переноса мы поставим
	задачу Коши $u(x,0)=v(x)$, где приняли $v(x)=x(x-h)$, то её решение
	\[u(x,0)=v(x-at)=(x-at)(x-at-h).\]

	Это значит, что если начальная функция -- парабола, то решение
	квадратично и по времени, и по пространству. Если её трижды
	продифференцировать, то получим 0, то есть схема второго порядка для
	параболы точная.

	Посмотрим, что на первом слою по времени.
	\[u(jh,\tau)=(jh-a\tau)(jh-a\tau-h)=h^2(j-r)(j-r-1).\]

	Это парабола, у которой нули в 0 и $h$ и чьи ветви направлены вверх. Она
	отрицательна на интервале от $(0,h)$.

	В операторной же форме
	\[u_j^1=\sum_{m\in M}C_m v_{j+m}^n.\]

	Все $C_m$ и $v_j$ неотрицательные, значит, $\forall j:u_j^1\ge 0$.

	Мы должны были получить то же самое, что и через открытую схему. Однако
	из неё мы можем получить следующее: если мы обозначим, $\alpha=j-r$, то
	верно неравенство
	\[\alpha(\alpha-1)<0\text{ при }\alpha\in(0,1)\quad\Rightarrow\quad
	r<j<1+r.\]

	И так как $r<1$, то при $j=1$ разностная схема одновременно
	неотрицательная и отрицательная, чего быть не может.
\end{proof}

\subsection{Устойчивость разностных схем}
Ещё одно из важных свойств, которое должна сохранять разностная схема -- это
устойчивость, то есть малым изменениям входным данным должно соответствовать
малые изменения решения.

Необходимо прогарантировать, что решение не ''развалится'' по мере прохождения
временных слоёв.

\begin{theorem}[спектральное условие устойчивости Неймана]
	Если разностное решение представить как $u_j^n=\lambda^ne^{ij\alpha}$,
	то разностная схема устойчива, если верно
	\[\forall\alpha\in\mathbb R\quad|\lambda(\alpha)|\le 1.\]
\end{theorem}

\begin{example}
	Исследуем устойчивость схемы по потоку \eqref{eq:CDE_forward_scheme}
	\[\frac{u_j^{n+1}-u_j^n}{\tau}+a\frac{u_{j+1}^n-u_j^n}{h}=0.\]

	Запишем схему в операторном виде:
	\[u_j^{n+1}=u_j^n-r(u_{j+1}^n-u_j^n).\]

	Теперь сделаем подстановку $u_j^n=\lambda^ne^{ij\alpha}$:
	\[\lambda^{n+1}e^{ij\alpha}=\lambda^ne^{ij\alpha}-r(\lambda^n
	e^{i(j+1)\alpha}-\lambda^ne^{ij\alpha})\quad\Leftrightarrow\quad
	\lambda(\alpha)=1+r-re^{i\alpha}.\]

	Геометрическое место точек $\lambda(\alpha)$ выглядит как окружность
	радиуса $r$ с центром в $(1+r,0)$:

	{\makeatletter
	\let\par\@@par
	\par\parshape0
	\everypar{}

	\begin{wrapfigure}{l}{0.5\textwidth}
		\subfile{graph-forward_scheme_stability}
	\end{wrapfigure}

	\leavevmode\\
	Легко увидеть, что почти всюду $|\lambda(\alpha)|>1$, поэтому
	разностная схема неустойчива при $\forall r>0$.
	\par}
\end{example}\leavevmode
\\

\begin{example}
	Исследуем устойчивость схемы Годунова \eqref{eq:godunov_scheme}
	\[\frac{u_j^{n+1}-u_j^n}{\tau}+a\frac{u_j^n-u_{j-1}^n}{h}=0.\]

	Проделаем те же шаги, что и в предыдущем примере, и получим
	\[\lambda(\alpha)=1-r+re^{-i\alpha}.\]
	
	\newpage
	Картина при $r\le 1$ выглядит вот так:
	{\makeatletter
	\let\par\@@par
	\par\parshape0
	\everypar{}

	\begin{wrapfigure}{l}{0.4\textwidth}
		\subfile{graph-godunov_scheme_stability}
	\end{wrapfigure}

	\leavevmode\\
	Таким образом, разностная схема устойчива при $r\le 1$. При $r=1$
	окружность $\lambda(\alpha)$ совпадёт с единичной, а при $r>1$
	перестанет в неё помещаться, что даёт неустойчивость при данных числах
	Куранта.\\\\
	\par}
\end{example}

\begin{example}
	Исследуем устойчивость схемы Лакса \eqref{eq:lax_scheme}
	\[\frac{u_j^{n+1}-\frac{u_{j-1}^n+u_{j+1}^n}{2}}{\tau}+
	a\frac{u_{j+1}^n-u_{j-1}^n}{2h}=0.\]

	В операторном виде схема имеет вид
	\[u_j^{n+1}=\frac{1}{2}\big((1+r)u_{j-1}^n+(1-r)u_{j+1}^n\big).\]

	Подставим $u_j^n=\lambda^ne^{ij\alpha}$ и выразим $\lambda$:
	\[\lambda(\alpha)=\cos\alpha+ir\sin\alpha.\]

	Графически это эллипс, но в этот раз мы решим неравенство алгебраическим
	методом:
	\[|\lambda|\le 1\quad\Leftrightarrow\quad\cos^2\alpha-r^2\sin^2\alpha\le
	1^2\quad\Leftrightarrow\quad(1-r^2)\sin^2\alpha\le 0.\]

	Из данного неравенства можно получить, что, чтобы разностная схема была
	устойчива при любом $\alpha$, необходимо и достаточно, чтобы $r\le 1$.
\end{example}

\subsubsection{Неявные схемы}
Особенность неявных схем в том, что их невозможно исследовать на монотонность по
Годунову. Однако их можно исследовать на устойчивость.

\begin{define}
	Разностная схема называется \textbf{безусловно устойчивой}, если она
	устойчива при любом положительном числе Куранта
	\eqref{eq:courant_number}.
\end{define}

Напомним, что число Куранта по определению не может быть отрицательным.

\begin{define}
	Разностная схема называется \textbf{условно устойчивой}, если она не
	безусловно устойчивая.
\end{define}

Все предыдущие схемы были условно устойчивыми, если были хоть как-нибудь
устойчивыми вообще.

\begin{example}
	Исследуем на устойчивость неявную разностную схему
	\[\frac{u_j^{n+1}-u_j^n}{\tau}+a\frac{u_{j+1}^{n+1}-u_{j-1}^{n+1}}{2h}=
	0.\]

	После подстановки $u_j^n=\lambda^ne^{ij\alpha}$ имеем
	\[\lambda=1-ir\lambda\sin\alpha\quad\Leftrightarrow\quad
	\lambda=\frac{1}{1-ir\sin\alpha}.\]

	Схема устойчива при любом $r$, так как модуль знаменателя не меньше
	единицы. Значит, схема безусловно устойчивая.

	Вот так выглядит шаблон этой разностной схемы: \\

	\subfile{graph-implicit_scheme}
\end{example}

Возникает резонный вопрос: а зачем использовать явные разностные схемы, если
есть неявные, которые безусловно устойчиые? Первые используются для
быстропротекающих процессов, таких как ударные волны, так как большой шаг по
времени может сильно испортить точность расчётов, да и явные схемы проще
реализовать. Для медленно протекающих процессов же, таких как паводковые течения
в реках, можно считать по неявным схемам, с использованием методом прогонки,
о котором поговорим в следующей главе. Такие схемы более ёмкие трудоёмкие.

\subsubsection{Схема-крест}
Рассмотрим трёхслойную по времени разностную схему.

\begin{define}
	\textbf{Схема-крест} имеет вид
	\[\frac{u_j^{n+1}-u_j^{n-1}}{\tau}+a\frac{u_{j+1}^n-u_{j-1}^n}{h}=
	0.\]
\end{define}

Шаблон схемы даёт объяснение названию: \\

\subfile{graph-cross_scheme}

\begin{example}
	Исследуем устойчивость схемы-крест. Сразу подставим
	$u_j^n=\lambda^ne^{ij\alpha}$ и преобразуем выражение:
	\[\lambda^2+2i\lambda\sin\alpha-1=0\quad\Leftrightarrow\quad
	\lambda_{1,2}=-ir\sin\alpha\pm\sqrt{1-r^2\sin^2\alpha}.\]

	При трёхслойных всегда будет по две лямбды. Рассмотрим несколько
	случаев значения числа Куранта:
	\begin{enumerate}[nosep]
		\item $r\le 1$: $|\lambda_{1,2}|^2=r^2\sin^2\alpha+1-
			r^2\sin^2\alpha=1$, что означает устойчивость схемы.
		\item $r>1$: Если подкоренное выражение неотрицательно, то
			имеем предыдущий случай, поэтому возьмём такое $\alpha$,
			что $1-r^2\sin^2\alpha < 0$. Тогда имеем
			\[|\lambda_1|^2=r\sin\alpha-\sqrt{r^2\sin^2\alpha-1},\]
			\[|\lambda_2|^2=r\sin\alpha+\sqrt{r^2\sin^2\alpha-1}.\]

			Если синус положительный, то модуль второго числа больше
			одного, если же он отрицательный, то модуль первого
			числа больше 1. Одной ветки достаточно для
			неустойчивости разностной схемы. Значит, при $r>1$ схема
			неустойчива.
	\end{enumerate}

	Несмотря на то что при $r\le 1$ схема устойчива, весь её спектр на
	единичной окружности, то есть с течением времени начальные данные будут
	без изменения. Поэтому такой схемой пользоваться нельзя.
\end{example}

\subsubsection{Устойчивость компактной разностной схемы}
\begin{define}
	Обозначим
	\[\Delta_x=\frac{T_h-T_{-h}}{2h},\quad A_x=\frac{T_h+4E+T_{-h}}{6},\]
	\[\Delta_t=\frac{T^\tau-T^{-\tau}}{2\tau},\quad A_t=\frac{T^\tau+4E+
	T^{-\tau}}{6},\]

	Тогда компактная и по времени, и по пространству разностная схема,
	аппроксимирующее линейное уравнение переноса
	\eqref{eq:convection-diffusion_equation}, имеет вид
	\[A_x\circ\Delta_t\circ u+aA_t\circ\Delta_x\circ u=0.\]
\end{define}

\begin{example}
	Исследуем данную компактную разностную схему на устойчивость. Запишем
	схему в операторной форме:
	\[(T_h+4E+T^{-h})(T^\tau-T^{-\tau})+r(T^\tau+4E+T^{-\tau})(T_h-T_{-h})
	=0.\]

	Сделаем подстановку $u_j^n=\lambda^ne^{ij\alpha}$ и сократим уравнение:
	\[\Big(\lambda-\frac{1}{\lambda}\Big)(e^{i\alpha}+4+e^{-i\alpha})+
	r\Big(\lambda+4+\frac{1}{\lambda}\Big)(e^{i\alpha}-e^{-i\alpha})=0.\]

	Избавимся от дробей и экспонент:
	\[(\lambda^2-1)(2+\cos\alpha)+ir(\lambda^2+4\lambda+1)\sin\alpha=0.\]

	Обозначим
	\[p=2+\cos\alpha,\quad q=r\sin\alpha,\quad z=p+iq.\]
	Уравнение примет вид
	\[z\lambda^2+4iq\lambda-\overline{z}=0,\]
	и его корни
	\[\lambda_{1,2}=\frac{-2iq\pm\sqrt{p^2-3q^2}}{z}.\]

	Как и в предыдущем примере, рассмотрим 2 случая относительно знака
	дискриминанта:
	\begin{enumerate}[nosep]
		\item $\forall \alpha\; D\ge 0$. Прежде чем как найти модули
			лямбд, найдём, при каких $r$ это достигается. Для этого
			применим формулу вспомогательного аргумента: обозначим
			$\rho=\sqrt{1+3r^2}$ и $\varphi$ так, что
			$\cos\varphi=\frac{1}{\rho}$,
			$\sin\varphi=\frac{\sqrt{3}r}{\rho}$:
			\[(p-\sqrt 3 q)(p+\sqrt3 q)\ge 0\quad\Leftrightarrow
			\quad\Big(\frac{2}{\rho}+\cos(\alpha+\varphi)\Big)
			\Big(\frac{2}{\rho}+\cos(\alpha-\varphi)\Big)\ge 0.\]

			Чтобы неравенство было верно при любом угле $\alpha$,
			необходимо и достаточно, чтобы $\frac{2}{\rho}\ge 1$.
			Из этого неравенства вытекает старое доброе условие
			Куранта $r\le 1$.

			Наконец, при таком условии модули лямбд всегда равны
			\[|\lambda_{1,2}|^2=\frac{p^2-3q^2+4q^2}{|z|^2}=1.\]

			Разностная схема при таком условии устойчивая. Однако,
			как и в случае со схемой-крестом, весь спектр на
			единичной окружности, что означает неизменчивость
			со временем значений.

		\item $\exists \alpha: D<0$. Это достигается при $r>1$.
			Чтобы было проще считать, прибавим 2:
			\[\lambda+2=\frac{2p+i\sqrt{3q^2-p^2}}{z},\quad
			|\lambda+2|=\sqrt 3.\]

			Эта окружность не лежит внутри единичной минимум из-за
			её большего радиуса. Это означает неустойчивость
			разностной схемы при $r>1$.
	\end{enumerate}
\end{example}

Чтобы по компактной схеме и схеме-кресту можно было считать, схемы
стабилизируют, однако мы этим заниматься не будем.

Разностные схемы имеют широкое применение в физике (см. уравнение
теплопроводности), в моделировании -- они буквально повсюду. На этой
ноте мы завершаем их столь долгое рассмотрение.

\end{document}
