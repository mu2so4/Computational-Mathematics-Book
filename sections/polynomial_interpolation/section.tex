\documentclass[../main.tex]{subfile}

\begin{document}

\section{Алгебраическое интерполирование}
Задача интерполирования функции состоит в том, чтобы по известным её значениям в
некоторых точках определить её значения в остальных точках области задания.
Такая задача возникает, например, когда по результатам измерения некоторой
физической величины в одних точках требуется определить её значения в других
точках или когда в целях ускорения вычислений желательно приблизить заданную
функцию более лёгкой в вычислении. Как правило, интерполируют полиномами.

\begin{define}\label{eq:interpolating_polynomial}
	Алгебраический полином $P_m(x)=\sum_{k=0}^{m}a_kx^k$ называется
	\textbf{интерполяционным} для функции $f(x)$, заданной на отрезке
	$[a,b]$ по её значениям $f(x_i)$ в $n+1$ попарно различных точках
	$x_i\in[a,b]$ (\textbf{узлах интерполяции}), если
	\[\forall i\in\{0,1,...,n\}\;P_m(x_i)=f(x_i).\]
\end{define}

\begin{define}\label{eq:interpolation_problem}
	\textbf{Задача алгебраической интерполяции} -- построить алгебраический
	полином, который удовлетворял бы условию
	\eqref{eq:interpolating_polynomial}. Далее будем писать просто ЗАИ.
\end{define}

Первый вопрос: а какие условия гарантируют существование и единственность
интерполяционного многочлена? Для $m+1$ неизвестного члена полинома мы имеем
$n+1$ условие. Единственность решения математических конечномерных задач
обычно обеспечивается равенством числа неизвестных числу условий. В противном
случае, мы можем либо получить несколько решений, либо не получить их вовсе.

\begin{theorem}[о существовании и единственности интерполяционного многочлена]
\label{eq:polynominal_theorem}
	ЗАИ при $n=m$ имеет единственное решение.
\end{theorem}

\begin{proof}
	Запишем систему в матричном виде:
	\[
	\begin{pmatrix}
		1	& x_0	& x_0^2	& ...	& x_0^n \\
		1	& x_1	& x_1^2	& ...	& x_1^n \\
		...	& ...	& ...	&\ddots	& ...	\\
		1	& x_n	& x_n^2	& ...	& x_n^n \\
	\end{pmatrix}
	\cdot
	\begin{pmatrix}
		a_0 \\
		a_1 \\
		... \\
		a_n \\
	\end{pmatrix}
	=
	\begin{pmatrix}
		f(x_0)	\\
		f(x_1)	\\
		...	\\
		f(x_n)	\\
	\end{pmatrix}
	\]

	Определитель этой матрицы -- определитель Вандермонда -- не равен нулю,
	так как $x_i\ne x_j$ при $i\ne j$. Это необходимо и достаточно для
	существования и единственности решения.
\end{proof}

\subsection{Интерполяционный многочлен Лагранжа}
В теореме \eqref{eq:polynominal_theorem}, мы доказали, что решение ЗАИ
существует, но там не было ни слова о том, как его искать. Считать методом
Крамера -- слишком долго, чтобы не наскучило. Один из способ определить
интерполяционный полином предложил Луи Лагранж.

\begin{theorem}[о представлении в форме Лагранжа]\label{eq:lagrange_polynomial}
	При $n=m$ решение ЗАИ \eqref{eq:interpolation_problem} представимо в
	\textbf{форме Лагранжа}:
	\[P_n(x)=\sum_{k=0}^{n}\frac{\omega(x)}{(x-x_k)\omega'(x_k)}f(x_k),\]
	где
	\[\omega(x)=\prod_{l=0}^{n}(x-x_l).\]
\end{theorem}

\begin{proof}
	Для каждого $k\in\{0,1,...,n\}$ рассмотрим частный случай ЗАИ:
	\[P_{n,k}(x_i)=\delta_{k,i},\;i\in\{0,1,...,n\},\]
	где $\delta_{k,i}$ -- символ Кронекера. Так как полином $P_{n,k}$
	степени $n$ по условию имеет $n$ корней $\{x_0, x_1, ..., x_{k-1},
	x_{k+1}, ..., x_n\}$, он может быть представлен в виде произведения
	мономов:
	\[P_{n,k}(x)=q_k\prod_{\substack{l=0\\ l\neq k}}^{n}(x-x_l),\]
	где $q_k$ берётся из условия $P_{n,k}(x_k)=1$:
	\[q_k=\frac{1}{\prod_{\substack{l=0\\ l\neq k}}^{n}(x-x_l)}.\]
	Обозначив $\omega(x)=\prod_{l=0}^{n}(x-x_l)$, перепишем многочлен:
	\[P_{n,k}(x)=\frac{\omega(x)}{(x-x_k)\omega'(x_k)}f(x_k).\]
	Очевидно, что линейная комбинация
	\[P_n(x)=\sum_{k=0}^{n}P_{n,k}(x)f(x_k)\]
	полиномов -- многочлен степени $n$. А так как $\forall i\in\{0,...,n\}$
	\[P_n(x_i)=\sum_{k=0}^{n}P_{n,k}(x_i)f(x_k)=\sum_{k=0}^{n}\delta_{i,k}
	f(x_k)=f(x_i),\]
	полином является интерполянтом функции $f(x)$.
\end{proof}

\begin{example}
	Найдём интерполяционный многочлен Лагранжа функции \\ $f(x)=\cos x +
	\frac{x}{2}$ на отрезке $[0,4]$. Вот список значений:\newline

	\begin{tabular}{*{4}{|c}|}
		\hline
		$x$	& 0	& 2	& 4	\\
		\hline
		$f(x)$	& 1	& 0.584	& 1.346	\\
		\hline
	\end{tabular}

	\[\omega(x)=x(x-2)(x-4).\]

	\[P_2(x)=\omega(x)\Big (\frac{1}{8x}-\frac{0.584}{4(x-2)}
	+\frac{1.346}{8(x-4)} \Big )=\]
	\[=0.125(x^2-6x+8)-0.146(x^2-4x)+0.168(x^2-2x)=\]
	\[=0.149x^2-0.502x+1.\]
	\newpage

	Найдём значения некоторых внутренних точек:\newline

	\begin{tabular}{*{3}{|c}|}
		\hline
		$x$     & 1     & 3     \\
		\hline
		$f(x)$  & 1.04	& 0.51	\\
		\hline
		$P_2(x)$& 0.645	& 0.818	\\
		\hline
	\end{tabular}
	\leavevmode\newline

	И изобразим всё это на графике:
	\newline

	\subfile{graph-lagrange_polynomial}

\end{example}

\subsection{Интерполяционный многочлен Ньютона}
Предположим, что мы по $n+1$ узлам интерполяции построили полином Лагранжа
$P_n(x)$ для функции $f(x)$. Затем нам стало известно новое значение
интерполяционной функции при $x_{n+1}$. Вопрос: как быстро мы сможем перестроить
полином под новый узел интерполяции? От полинома $P_n(x)$ построить $P_{n+1}(x)$
нереально сложно, так как из \eqref{eq:lagrange_polynomial} придётся пересчитать
все ''коэффициенты'' вида \[\frac{\omega(x)}{(x-x_k)\omega'(x_k)}.\]

Надо найти новую запись интерполяционного многочлена. Заметьте -- именно запись,
так как по теореме \eqref{eq:polynominal_theorem} полином единственнен.

\begin{define}\label{eq:divided_difference}
	\textbf{Разделённая разность $n$-го порядка} функции $f(x)$ на попарно
	разных узлах $x_0,...,x_n$ -- это число
	\[f[x_0,..., x_{n}]=\frac{f[x_1,...,x_n]-f[x_0,...,x_{n-1}]}
	{x_n-x_0},\]
	где $f[y_0,...,y_{n-1}]$ -- разделённая разность $n-1$ порядка, а
	разделённая разность первого порядка равна
	\[\boxed{f[x_0,x_1]=\frac{f(x_1)-f(x_0)}{x_1-x_0}}.\]
\end{define}

\begin{lemma}\label{eq:div_diff_formula}
	Для разделённой разности $n$-го порядка справедлива формула
	\[f[x_0,...,x_n]=\sum_{i=0}^{n}\frac{f(x_i)}
	{\prod_{\substack{j=0\\ j\neq i}}^{n}(x_i-x_j)}.\]
\end{lemma}

\begin{proof}
	Доказывать мы будем методом математической индукции. Для $n=1$ верность
	леммы очевидна для разделённой разности первого порядка.

	Теперь положим, что лемма верна при $n=m$. Докажем лемму для $m+1$.
	\[f[x_0,...,x_{m+1}]=\frac{f[x_1,...,x_{m+1}]-f[x_0,...,x_m]}
	{x_{m+1}-x_0}=\]
	\[{\overset{\substack{\text{инд.}\\\text{предп.}}}{=\joinrel=}}
	\frac{1}{x_{m+1}-x_0}\Big( \sum_{i=1}^{m+1}\frac{f(x_i)}
	{\prod_{\substack{j=1\\ j\neq i}}^{m+1}(x_i-x_j)} - \sum_{k=0}^{m}
	\frac{f(x_k)}{\prod_{\substack{l=0\\ k\neq l}}^{m}(x_k-x_l)}\Big)=\]
	\[=\frac{f(x_0)}{\prod_{i=1}^{m+1}(x_0-x_i)} + \frac{1}{x_{m+1}-x_0}
	\sum_{i=1}^{m} \Big(\frac{f(x_i)}{\prod_{\substack{j=1\\ j\neq i}}^
	{m+1}(x_i-x_j)}-\]
	\[-\frac{f(x_i)}{\prod_{\substack{j=0\\ j\neq i}}^{m}(x_i-x_j)}
	\Big) + \frac{f(x_{m+1})}{\prod_{i=0}^{m}(x_{m+1}-x_i)}. \tag{*}\]

	Преобразуем выражение, которое стоит внутри оператора суммы:
	\[\frac{f(x_i)}{\prod_{\substack{j=1\\ j\neq i}}^{m+1}(x_i-x_j)}-
	\frac{f(x_i)}{\prod_{\substack{j=0\\ j\neq i}}^{m}(x_i-x_j)}=\]
	\[=\frac{f(x_i)\big((\cancel{x_i}-x_0)-(\cancel{x_i}-x_{m+1})\big)}
	{\prod_{\substack{j=0\\ j\neq i}}^{m+1}(x_i-x_j)}=
	\frac{f(x_i)(x_{m+1}-x_0)}{\prod_{\substack{j=0\\ j\neq i}}^{m+1}
	(x_i-x_j)},\]

	И окончательно из (*) получим:
	\[f[x_0,...,x_n]=\frac{f(x_0)}{\prod_{i=1}^{m+1}(x_0-x_i)}+\sum_{j=1}^
	{m} \frac{f(x_j)}{\prod_{\substack{i=0\\ i\neq j}}^{m+1}(x_j-x_i)}+\]
	\[+\frac{f(x_{m+1})}{\prod_{i=0}^{m}(x_{m+1}-x_i)}=
	\sum_{i=0}^{m+1}\frac{f(x_i)}{\prod_{\substack{j=0\\ j\neq i}}^{m+1}
	(x_i-x_j)}.\]
\end{proof}

\begin{lemma}\label{eq:next_int_pol}
	Пусть $P_k(x)$ и $P_{k+1}(x)$ -- интерполяционные многочлены функции
	$f(x)$, тогда для них верно равенство:
	\[P_{k+1}(x)-P_k(x)=A_{k+1}\prod_{i=0}^{k}(x-x_k),\]
	где
	\[A_{k+1}=\frac{f(x_{k+1})-P_k(x_{k+1})}{\prod_{i=0}^{k}(x_{k+1}-x_i)}.\]
\end{lemma}

\begin{proof}
	Очевидно, что разность $P_{k+1}(x)-P_{k}(x)$ -- полином степени $k+1$.
	Также верно, что для первых $k+1$ интерполяционных узлов $P_{k+1}(x_i)=
	P_{k}(x_i)$. Поскольку у полинома степени $k+1$ ровно столько же нулей,
	верно разложение
	\[P_{k+1}(x)-P_{k}(x)=A_{k+1}\prod_{i=0}^{k}(x-x_i),\]
	где $A_{k+1}$ -- некоторая константа, которую можно найти из представления
	\[\underset{=f(x_{k+1})}{\underbrace{P_{k+1}(x_{k+1})}}-P_{k}(x_{k+1})=
	A_{k+1}\prod_{i=0}^{k}(x_{k+1}-x_i)\Rightarrow\]
	\[\Rightarrow \boxed{A_{k+1}=\frac{f(x_{k+1})-P_k(x_{k+1})}
	{\prod_{i=0}^{k}(x_{k+1}-x_i)}}.\]
\end{proof}

\begin{lemma}\label{eq:newton_koef_formula}
	Коэффициенты $A_k$ из леммы \eqref{eq:next_int_pol} равны разделённым
	разностям $k$-го порядка, то есть $A_k=f[x_0,...,x_k].$

\end{lemma}

\begin{proof}
	По формуле
	\[A_k=\frac{f(x_k)-P_{k-1}(x_k)}{\prod_{i=0}^{k-1}(x_k-x_i)}=
	\frac{f(x_k)}{\prod_{i=0}^{k-1}(x_k-x_i)} - \frac{P_{k-1}(x_k)}
	{\prod_{i=0}^{k-1}(x_k-x_i)}.\tag{*}\]

	Во втором слагаемом запишем полином $P_{k-1}(x_k)$ в форме\\Лагранжа
	\eqref{eq:lagrange_polynomial}:
	\[P_{k-1}(x)=\sum_{i=0}^{k-1}\frac{\omega(x)}{(x-x_i)\omega'(x_i)}
	f(x_i),\;\;
	\omega(x)=\prod_{j=0}^{k-1}(x-x_j).\]

	Заметим, что знаменатели в $(*)$ равны $\omega(x_k)$:
	\[A_k=\frac{f(x_k)}{\omega(x_k)} - \frac{1}{\cancel{\omega(x_k)}}
	\sum_{i=0}^{k-1}\frac{\cancel{\omega(x_k)}}{(x_k-x_i)\omega'(x_k)}
	f(x_i).\]
	И с учётом
	$\omega'(x_i)=\mathlarger\prod_{j=0,\;j\ne i}^{k-1}(x_i-x_j)$
	окончательно получаем:
	\[A_k=\frac{f(x_k)}{\prod_{j=0}^{k-1}(x_k-x_j)} +
	\sum_{i=0}^{k-1}\frac{f(x_i)}{\prod_{\substack{j=0\\j\ne i}}^{k-1}
	(x_i-x_j)}=\sum_{i=0}^{k}\frac{f(x_i)}{\prod_{\substack{j=0\\j\ne i}}^
	{k}(x_i-x_j)},\]
	что по лемме \eqref{eq:div_diff_formula} и является разделённой
	разностью $k$-го порядка на попарно разных узлах $x_0,...,x_k$.
\end{proof}

\begin{theorem}[о представлении в форме Ньютона]\label{eq:newton_polynomial}
	При $n=m$ решение ЗАИ \eqref{eq:interpolation_problem} представимо в
	\textbf{форме Ньютона}:
	\[\boxed{P_n(x)=\sum_{k=0}^{n}\big(f[x_0,...,x_k]\prod_{i=0}^{k-1}
	(x-x_i)\big)},\]
	где $f[x_0,...,x_k]$ -- разделённая разность $k$-го порядка
	\eqref{eq:divided_difference}.
\end{theorem}

\begin{proof}
	Данная теорема является следствием лемм \eqref{eq:next_int_pol} и
	\eqref{eq:newton_koef_formula}:
	\[P_n(x)=P_{n-1}(x)+A_n\prod_{k=0}^{n-1}(x-x_k)=...=
	\sum_{k=0}^{n}\big(A_k\prod_{i=0}^{k-1}(x-x_i)\big),\]
	в которой $A_k=f[x_0,...,x_k]$ и $A_0=f(x_0)$.
\end{proof}

\begin{example}\label{eq:newton_polynomial_example}
	Найдём для функции $f(x)=e^x-2\sin x$ на отрезке $[-4,2]$
	интерполяционный многочлен в форме Ньютона по следующей таблице:\\

	\begin{tabular}{*{5}{|c}|}
		\hline
		$x$	& -4	& -2	& 0	& 2	\\
		\hline
		$f(x)$	&-1.495	& 1.954	& 1.000	& 5.570	\\
		\hline
	\end{tabular}\leavevmode\newline

	Многочлен будет в форме
	\[P_3(x)=f(-4)+f[-4,-2](x+4)+f[-4,-2,0](x+4)(x+2)+\]
	\[+f[-4,-2,0,2]\big(x(x+4)(x+2)\big).\]

	Найдём разделённые разности:

	\begin{table}[h]
		\centering
		\resizebox{\columnwidth}{!}{\begin{tabular}{*{4}{l}}
			$f(-4)=\boxed{-1.495}$ & & & \\
			 & $f[-4,-2]=\boxed{1.725}$ & & \\
			$f(-2)=1.954$ & & $f[-4,-2,0]=\boxed{-0.551}$ & \\
			 & $f[-2,0]=-0.477$ & & $f[-4,-2,0,2]=\boxed{0.207}$ \\
			$f(0)=1.000$ & & $f[-2,0,2]=0.691$ & \\
			 & $f[0,2]=2.285$ & & \\
			$f(2)=5.570$ & & & \\
		\end{tabular}}
	\end{table}
	Досчитаем полином:
	\[P_3(x)=-1.495+1.725(x+4)-0.551(x^2+6x+8)+0.207(x^3+6x^2+8x)=\]
	\[=0.207x^3+0.691x^2+0.075x+1.\]
	\newpage

	Отобразим исходную функцию и полученный полином на графике:

	\subfile{graph-newton_polynomial}

\end{example}

\subsection{Оценка погрешности интерполирования}
Четвёртый и самый главный вопрос интерполирования -- насколько отличаются
значения интерполяционного многочлена $P_n(x)$ от значений интерполируемой
функции $f(x)$? Сначала дадим простое определение.

\begin{define}\label{eq:interpolation_error}
	\textbf{Погрешность} или \textbf{ошибка интерполирования} функции $f(x)$
	по её значениям $f(x_i)$ в попарно различных точках $x_0,...,x_n$
	интервала $[a,b]$ интерполяционным многочленом $P_n(x)$ -- это их
	разность $\boxed{R_n(x)=f(x)-P_n(x)}$.
\end{define}

То, что в узлах интерполяции погрешности нет -- это понятно. Нам бы хотелось
найти погрешность и в других точках.

\begin{lemma}\label{eq:interpolation_error_form}
	Если $y\in[a,b]\backslash\{x_0,...,x_n\}$, то
	\[R_n(y)=f[x_0,...,x_n,y]\omega(y),\quad
	\omega(y)=\prod_{k=0}^{n}(y-x_n).\]
\end{lemma}

\begin{proof}
	Построим новый интерполяционный полином $P_{n+1}(x)$ для $f(x)$ на
	$n+2$ попарно различных узлах $\{x_0,...,x_n,y\}$. Представим его в
	форме Ньютона \eqref{eq:newton_polynomial}:
	\[P_{n+1}(x)=P_n(x)+f[x_0,...,x_n,y]\omega(x)\Rightarrow\]
	\[\Rightarrow P_n(x)=P_{n+1}(x)-f[x_0,...,x_n,y]\omega(x).\]
	И тогда
	\[R_n(y)=f(y)-P_n(y)=\cancel{f(y)}-\cancel{P_{n+1}(y)}+f[x_0,...,x_n,y]
	\omega(y)\Rightarrow\]
	\[\Rightarrow \boxed{R_n(y)=f[x_0,...,x_n,y]\omega(y)}.\]
\end{proof}

\begin{lemma}\label{eq:xi_in_ab}
	Если $f(x)\in C^n([a,b])$, то
	\[\exists\xi\in[a,b]:f[x_0,...,x_n] = \frac{f^{(n)}(\xi)}{n!}.\]
\end{lemma}

\begin{proof}
	Рассмотрим погрешность интерполирования $R_n(x)=f(x)-P_n(x)$. Выразим
	$P_n(x)$ в форме Ньютона \eqref{eq:newton_polynomial} и
	продифференцируем $n$ раз:
	\[R_n^{(n)}(x) = \Big(f(x) - \sum_{k=0}^{n}\big(f[x_0,...,x_k]\prod_
	{i=0}^{k-1}(x-x_i)\big)\Big)^{(n)}=\]
	\[=f^{(n)}(x) - n!f[x_0,...,x_n].\]

	Теперь понятно, что, чтобы доказать лемму, нужно доказать
	существование нуля функции $R_n^{(n)}(x)$.

	Оценим количество нулей функции $R_n(x)$. Их не меньше $n+1$, так как
	по определению \eqref{eq:interpolating_polynomial}
	\[\forall i\in \{0,1,...,n\}\quad f(x_i)=P_n(x_i).\]
	Не уменьшая общности, будем считать, что $x_0<x_1<...<x_n$.
	По теореме Ролля, на интервалах $(x_0,x_1),...,(x_{n-1},x_n)$ содержится
	минимум по одному нулю производной $R'_n(x)$, таким образом, на отрезке
	$[a,b]$ их менее $n$ попарно различных корней.

	Провернём это же действие уже с функцией $R'_n(x)$ и получим, что у
	$R''_n(x)$ не менее $n-1$ попарно различных корней. Сделав так $n$ раз,
	получим, что у функции $R_n^{(n)}(x)$ есть минимум один ноль на $[a,b]$.
\end{proof}

\begin{theorem}\label{eq:interpolation_der_error_form}
	Если $f(x)\in C^{n+1}([a,b])$, то погрешность интерполирования функции
	$f(x)$ по её значениям $f(x_i)$ в попарно различных точках $x_0,...,x_n$
	отрезка $[a,b]$ интерполяционным полиномом $P_n(x)$ может быть
	представлена в следующем виде:
	\[\boxed{R_n(y)=\frac{f^{(n+1)}(\xi)}{(n+1)!}\omega(y)},\quad\xi\in
	[a,b],\;\omega(y)=\prod_{k=0}^{n}(y-x_k).\]
\end{theorem}

\begin{proof}
	Если $y$ совпадает с одним из узлов интерполяции, то теорема работает
	$\forall\xi\in[a,b]$, так как $\omega(y)=0$.

	В противном случае, по леммам \eqref{eq:interpolation_error_form} и
	\eqref{eq:xi_in_ab}
	\[R_n(y)=f[x_0,...,x_n,y]\omega(y)=\frac{f^{(n+1)}(\xi)}{(n+1)!}
	\omega(y).\]
\end{proof}

Иногда является полезным найти не точное значение погрешности, а её верхнюю
оценку, особенно если учесть, что поиск параметра $\xi$ может оказаться
затруднительным.

\begin{define}\label{eq:function_norm}
	\textbf{Норма функции} $f(x)$ на непрерывном отрезке $[a,b]$ -- это
	число вида
	\[||f(x)||_{C[a,b]}=\underset{x\in[a,b]}{max}|f(x)|.\]
\end{define}

\begin{corollary}[оценка погрешности интерполяции]
	Если $f(x)\in C^{n+1}([a,b])$, то верно неравенство
	\[||R_n(x)||_{C[a,b]}\le \frac{||f^{(n+1)}(x)||_{C[a,b]}}{(n+1)!}
	||\omega(x)||_{C[a,b]},\quad \omega(x)=\prod_{k=0}^{n}(x-x_k).\]
\end{corollary}

\begin{proof}
	Из теоремы \eqref{eq:interpolation_der_error_form} верно равенство
	\[|R_n(y)|=\frac{|f^{(n+1)}(\xi)|}{(n+1)!}|\omega(y)|.\]
	Оценим равенство сверху:
	\[|R_n(y)|\le\frac{||f^{(n+1)}(x)||_{C[a,b]}}{(n+1)!}||\omega(x)||_
	{C[a,b]};\]
	поскольку это верно $\forall y\in[a,b]$, можем заключить, что
	\[||R_n(x)||_{C[a,b]}\le \frac{||f^{(n+1)}(x)||_{C[a,b]}}{(n+1)!}
	||\omega(x)||_{C[a,b]}.\]
\end{proof}

Узлы интерполяции следует выбирать как можно равномерней, иначе на участке
отрезка, обеднённом ими, погрешность будет гораздо выше.

\begin{example}
	Оценим погрешность интерполяционного полинома функции $f(x)=e^x-2\sin x$
	из примера \eqref{eq:newton_polynomial_example}, на том
	же отрезке $[-4,2]$ и на тех же узлах $\{-4,-2,0,2\}$.

	Начнём с $\omega(x)=(x^2-4)(x^2+4x)=x^4+4x^3-8x^2-16x$. Попробуем найти
	экстремумы этого многочлена:
	\[\omega'(x)=4x^3+12x^2-8x-16=4(x+1)(x+1+\sqrt 5)(x+1-\sqrt 5).\]

	Так как все эти точки одинарной кратности, они являются экстремумами.
	Более того, все они принадлежат отрезку $[-4,2]$. Проверим все из них
	(на краях отрезка функция зануляется):
	\[\omega(-1)=9,\;\omega(-1\pm\sqrt 5)=-16\Rightarrow
	||\omega(x)||_{C[-4,2]}=16.\]

	Точно найти максимум модуля функции $f^{(4)}(x)=e^x-2\sin x$ не
	получится, поэтому попробуем его оценить. Нижняя оценка очевидна:
	$f^{(4)}(x)>-2$. Верхняя оценка, очевидно, достигается в точке 2:
	$f^{(4)}(2)=e^2-\sin 2\le 7.4-2\cdot\frac{\sqrt 3}{2}\le 7.4-1.7=5.7$,
	тогда
	\[||R_3(x)||_{C[-4,2]}\le\frac{||e^x-2\sin x||_{C[-4,2]}}{4!}
	||\omega(x)||_{C[-4,2]}\le\frac{5.7\cdot 16}{24}=3.8.\]
	На практике, эта оценка даже близко не была достигнута.
\end{example}

\end{document}
