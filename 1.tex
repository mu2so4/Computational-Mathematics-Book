\documentclass[main.tex]{subfile}
\usepackage{cmap}
\usepackage{textbook}

\begin{document}

\section{Численное решение алгебраических уравнений}

\begin{define}
	\textbf{Метрическим пространством} называется пространство $M$, в
	котором определена операция метрики $\rho: M^2 \rightarrow \mathbb R $
	такая, что $\forall x, y, z \in M$ верно
	\begin{enumerate}
		\item $\rho(x,y)\ge 0$, причём $\rho(x,y) = 0 \Leftrightarrow x=y$;
		\item $\rho(x,y)=\rho(y,x)$;
		\item $\rho(x,y)\le \rho(x,z) + \rho(z,y)$.
	\end{enumerate}
\end{define}

\begin{define}
	\textbf{Сжимающим оператором} на метрическом пространстве $M$ с метрикой
	$\rho$ называется оператор $f: M \rightarrow M$ такой, что\\
	$\exists q \in(0;1):\forall x,y \in M\;\; \rho(f(x),f(y)) \le q \rho(x,y)$.
\end{define}

\end{document}
