\documentclass[main.tex]{subfile}
\usepackage{cmap}
\usepackage{textbook}

\begin{document}

\section{Численное решение алгебраических уравнений}

\begin{define}
	\textbf{Метрическим пространством} называется пространство $M$, в
	котором определена операция метрики $\rho: M^2 \rightarrow \mathbb R $
	такая, что $\forall x, y, z \in M$ верно
	\begin{enumerate}
		\item $\rho(x,y)\ge 0$, причём $\rho(x,y) = 0 \Leftrightarrow x=y$;
		\item $\rho(x,y)=\rho(y,x)$;
		\item $\rho(x,y)\le \rho(x,z) + \rho(z,y)$.
	\end{enumerate}
\end{define}

\begin{define}
	\textbf{Сжимающим оператором} на метрическом пространстве $M$ с метрикой
	$\rho$ называется оператор $f: M \rightarrow M$ такой, что\\
	$\exists q \in(0;1):\forall x,y \in M\;\; \rho(f(x),f(y)) \le q \rho(x,y)$.

	Число $q$ назовём \textbf{коэффициентом сжатия}.
\end{define}

\begin{theorem}[о сжимающем операторе]
	Пусть $M$ -- метрическое пространство с метрикой $\rho$, $f$ -- сжимающий
	оператор на $M$ с коэффициентом сжатия $q$. Тогда:
	\begin{enumerate}
		\item $\exists!\;\widetilde{x}\in M: f(\widetilde{x})=\widetilde{x}$;
		\item Любая последовательность $\{x_n\}$ такая, что $x_0 \in M$ --
			произвольный, $x_{k+1}=f(x_k)$, сходится к $\widetilde{x}$;
		\item Для всякой такой последовательности верно\\
			$\rho(x_n, \widetilde{x}) \le q ^ n \rho(x_0, \widetilde{x})$.
	\end{enumerate}
\end{theorem}

Данная теорема была доказана в курсе "Дополнительные главы математического анализа"{}.

\end{document}
