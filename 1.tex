\documentclass[main.tex]{subfile}
\usepackage{cmap}
\usepackage{textbook}

\begin{document}

\section{Численное решение алгебраических уравнений}

Предположим, что нам нужно решить в действительных числах уравнение
$y(x)=0$. Если это просто абстрактное уравнение, то нет единого
алгоритма, как его решить. Первое, что можно сделать -- это локализовать
корни, но это можно сделать лишь по виду функции.

\subsection{Метод бисекции}

Простейший метод отыскания нуля функции -- это метод бисекции, или метод
деления отрезка пополам. Для него нужна лишь совсем небольшая математическая
база.

\begin{algorithm}[метод бисекции]
	Пусть для функции $f(x)$ нам известно, что
	\begin{enumerate}
		\item $f(x)\in C([a,b])$;
		\item $f(a)f(b)<0$, то есть на концах отрезка функция
			принимает значения, противоположные по знаку.
	\end{enumerate}

	Организуем систему вложенных отрезков $[a_n, b_n]$ так, что\\
	$[a_1, b_1]=[a,b]$, а следующие отрезки сформированы следующим образом:
	если $c_k=\frac{a_k+b_k}{2}$, то:
	\begin{itemize}
		\item Если $f(c_k)=0$, то мы нашли корень, можно не продолжать;
		\item В противном случае, мы берём тот конец отрезка, функция от
			которой не совпала по знаку с $f(c_k)$, и формируем из него
			и точки $c_k$ отрезок $[a_{k+1},b_{k+1}]$.
	\end{itemize}

	Условие выхода: $|f(c_n)|<\varepsilon$, где $\varepsilon>0$ -- некоторая
	погрешность.

	Если данный алгоритм усилить монотонным возрастанием или убыванием на $[a,b]$,
	то ноль на нём будет единственный.
\end{algorithm}

Корректность данного алгоритма напрямую следует из теоремы Больцано-Коши, которая
и доказывается методом бисекции в сочетании с теоремой о двух милиционерах.


\subsection{Итерационные методы решения алгебраических уравнений}

Метод бисекции гарантированно даёт нам результат. Но нам бы хотелось иметь
более быструю сходимость.

\begin{define}
	\textbf{Метрическим пространством} называется пространство $M$, в
	котором определена операция метрики $\rho: M^2 \rightarrow \mathbb R $
	такая, что $\forall x, y, z \in M$ верно
	\begin{enumerate}
		\item $\rho(x,y)\ge 0$, причём $\rho(x,y) = 0 \Leftrightarrow x=y$;
		\item $\rho(x,y)=\rho(y,x)$;
		\item $\rho(x,y)\le \rho(x,z) + \rho(z,y)$.
	\end{enumerate}

	Если дополнительно верно, что каждая фундаментальная последовательность
	$M$ сходится к элементу, принадлежащему ему, то $M$ --
	\textbf{полное метрическое пространство}.
\end{define}

\begin{define}
	\textbf{Сжимающим оператором} на метрическом пространстве $M$ с метрикой
	$\rho$ называется оператор $f: M \rightarrow M$ такой, что\\
	$\exists q \in(0;1):\forall x,y \in M\;\; \rho(f(x),f(y)) \le q \rho(x,y)$.

	Число $q$ назовём \textbf{коэффициентом сжатия}.
\end{define}

\begin{theorem}[о сжимающем операторе]
	Пусть $(M,\rho)$ -- полное метрическое пространство, $f$ -- сжимающий
	оператор на $M$ с коэффициентом сжатия $q$. Тогда:
	\begin{enumerate}
		\item $\exists!\;\widetilde{x}\in M: f(\widetilde{x})=\widetilde{x}$;
		\item Любая последовательность $\{x_n\}$ такая, что $x_0 \in M$ --
			произвольный, $x_{k+1}=f(x_k)$, сходится к $\widetilde{x}$;
		\item Для всякой такой последовательности верно\\
			$\rho(x_n, \widetilde{x}) \le q ^ n \rho(x_0, \widetilde{x})$.
	\end{enumerate}
\end{theorem}

Данная теорема была доказана в курсе "Дополнительные главы математического анализа"{}.
В литературе эта теорема также известна как "Теорема Банаха о неподвижной точке"{}.

Всё это нужно было для того, чтобы доказать красивую, но практически бесполезную
теорему.

\begin{theorem}[о сходимости итерационного процесса]
	Пусть нам необходимо найти неподвижную точку функции $g(x)$. Зададим
	в качестве полного метрического пространства отрезок $G=\{x: |x-a| \le r\}$,
	где $a$ и $r$ -- параметры отрезка. Для того чтобы итерационный процесс
	$x_{k+1}=g(x_k)$ сошёлся, достаточно, чтобы:
	\begin{enumerate}
		\item $\forall x,y\in G\;|g(x)-g(y)|\le q|x-y|$, где $q\in(0,1)$ --
			некоторая константа, то есть верно условие Липшица;
		\item $|g(a)-a|\le(1-q)r$.
	\end{enumerate}
\end{theorem}

\beginproof

	Нам всего лишь нужно доказать, что $g(x)$ замкнута на $G$.
	\begin{multline*}
		\forall x\in G:|g(x)-a|=|g(x)-g(a)+g(a)-a|\le \\
		\le|g(x)-g(a)|+|g(a)-a|.
	\end{multline*}

	Применив условия 1 и 2 к первому и второму модулю соответственно, получаем
	\begin{align*}
		|g(x)-g(a)|+|g(a)-a|\le q|x-a|+(1-q)r\le qr+r-qr=r.
	\end{align*}

	Следовательно, $\forall x\in G: |g(x)-a|\le r\Rightarrow\forall x\in G\;\;g(x)\in G$,
	что и даёт нам замкнутость $g(x)$ на этом отрезке. Условие 1 и замкнутость
	даёт нам право считать $g(x)$ сжимающим отображением, поэтому тут применима
	теорема Банаха о неподвижной точке -- итерационный процесс сходится.\qed

\end{document}
