\documentclass{article}

\begin{document}
\unnumberedsection{Список изменений}

\begin{itemize}[nosep]
\item Версия 0.12 от 31.12.2022 (текущая версия):
	\begin{enumerate}[nosep]
		\item Исправлены коэффициенты в доказательстве теоремы
			\eqref{eq:boundary_problem_condition};
		\item Было обозначено, что \eqref{eq:godunov_scheme} -- это
			схема Годунова;
		\item Исправлена опечатка в примере
			\eqref{eq:boundary_problem_example};
		\item Добавлена 12 лекция с разностными схемами Лакса и
			Лакса-Вендроффа и запретом Годунова.
	\end{enumerate}

\item Версия 0.11 от 30.12.2022:
	\begin{enumerate}[nosep]
		\item Устранена неоднозначность с обозначением векторов;
		\item Добавлены недостающие определения в параграф
			''Линейные гиперболические системы'';
		\item Исправлена формула из введения в ''Методы Рунге-Кутты'';
		\item Добавлена 11 лекция с начально-краевой задачей
			гиперболической системы и разностными схемами,
			аппроксимирующие линейное уравнение переноса.
	\end{enumerate}

\item Версия 0.10 от 18.12.2022:
	\begin{enumerate}[nosep]
		\item Внесены правки в формулу коэффициентов из
			теоремы
			\eqref{eq:compact_difference_scheme_theorem};
		\item Исправлены опечатки в пунктах
			\eqref{eq:difference_consequence},
			\eqref{eq:complex_roots_de}
			и в примерах
			\eqref{eq:differential_equation_simplest_example},
			\eqref{eq:compact_ds_example},
			\eqref{eq:spoiled_ds_example},
			\eqref{eq:4dot_difference_equation},
			\eqref{eq:complexes_difference_eq};
		\item Добавлена 10 лекция с численным решением нелинейных
			обыкновенных дифференциальных уравнений и уравнениями
			в частных производных;
		\item Добавлен раздел ''Список изменений'', то есть добавлен
			данный раздел.
	\end{enumerate}

\item Версия 0.9 от 04.12.2022:
	\begin{enumerate}[nosep]
		\item Исправлены опечатки в пунктах
			\eqref{eq:composite_simpsons_3_8_rule} и
			\eqref{eq:compact_difference_scheme_theorem};
		\item Обновлено обоснование разрешимости системы уравнений из
			примера \eqref{eq:central_do_example};
		\item Добавлена 9 лекция с теорией по решению линейных
			разностных уравнений с постоянными коэффициентами и
			примером неустойчивой разностной схемы.
	\end{enumerate}

\item Версия 0.8 от 08.11.2022:
	\begin{enumerate}[nosep]
		\item Добавлена 8 лекция с двух- и трёхточечными, а также с
			составными разностными схемами;
		\item Введены заголовки 3 уровня;
		\item Переименованы заголовки 1 уровня.
	\end{enumerate}
\end{itemize}
\end{document}
