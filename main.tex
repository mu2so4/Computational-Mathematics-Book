\documentclass[17pt, a4paper]{extarticle}
\usepackage{cmap}
\usepackage{bookmark}
\usepackage{amsmath}
\usepackage{textbook}

\newcommand{\unnumberedsection}[1]{\addcontentsline{toc}{section}{#1} \section*{#1}}

\begin{document}

\begin{titlepage}
	\null\vfill

	\begin{center}
		{\large Муратов М. А.}
		\vskip 1cm

		{\huge\MakeUppercase{Вычислительная математика}}
		\vskip 2cm

		{\large Студенческое учебное пособие на\\основе лекций В. В. Остапенко}
	

		\vfill
		\vfill
		ФИТ НГУ
		\vskip 2 cm

		Новосибирск\\2022
	\end{center}
	\hfill
\end{titlepage}

\setcounter{page}{2}

\renewcommand{\contentsname}{Оглавление}
\tableofcontents

\newpage

\unnumberedsection{Предисловие}

Если вы нашли в данном учебном пособии опечатку или неточности, просьба об этом
сообщить \href{https://vk.com/id212526330}{\underline{автору}} данного пособия.
Тем, кто хочет в пособие внести серьёзные правки, стоит посетить
\href{https://github.com/mu2so4/Computational-Mathematics-Book}
{\underline{репозиторий на Github'е}} и ознакомиться с условиями их внесения.

\newpage

\section{Пробный заголовок}


\subsection{Пробный подзаголовок}

Some text

Какой-то текст
\begin{define}
Грань, у которой все стороны граничные, называется внешней.
\end{define}

\begin{define}
Замкнутый терм -- терм без свободных переменных.
\end{define}

\begin{lemma}
Бла-бла-бла.
\end{lemma}

\begin{theorem}[о наилучшей оценке]
	Не существует наилучшей оценки.
\end{theorem}

\noproof

\begin{remarkthm}
Это всё ложь!
\end{remarkthm}

\proofstart

To prove it by contradiction try and assume that the statement is false,
proceed from there and at some point you will arrive to a contradiction. \qed

\subsection{Что-то большое}

\begin{define}
Sed ut perspiciatis unde omnis iste natus error sit voluptatem accusantium doloremque laudantium, totam rem aperiam, eaque ipsa quae ab illo inventore
veritatis et quasi architecto beatae vitae dicta sunt explicabo.

Nemo enim ipsam voluptatem quia voluptas sit aspernatur aut odit aut fugit, sed quia
consequuntur magni dolores eos qui ratione voluptatem sequi nesciunt. Neque porro quisquam est, qui dolorem ipsum quia dolor sit amet, consectetur,
adipisci velit, sed quia non numquam eius modi tempora incidunt ut labore et dolore magnam aliquam quaerat voluptatem.

Ut enim ad minima veniam, quis
nostrum exercitationem ullam corporis suscipit laboriosam, nisi ut aliquid ex ea commodi consequatur? Quis autem vel eum iure reprehenderit qui in ea
voluptate velit esse quam nihil molestiae consequatur, vel illum qui dolorem eum fugiat quo voluptas nulla pariatur?
\end{define}

\begin{define}
Грань, у которой все стороны граничные, называется внешней.

e

u

o

e

u

a

v

j
\end{define}

\begin{define}
Замкнутый терм - терм без свободных переменных.

\end{define}

\newpage

\section{Ого!!!}

\subsection{Ну быть такого не может!}

This message means that a line of your document is too long to fit within the
horizontal space on the page, and TeX couldn't find a good way to break
it apart. This will usually result in text hanging out past the margin, possibly
even running off the side of the page.

\begin{define}
Грань, у которой все стороны граничные, называется внешней.
\end{define}

\begin{define}
Замкнутый терм - терм без свободных переменных.
\end{define}

\begin{example}
ukieukejukjkjuk
\end{example}

\newpage

\subsection{Русские}

Равным образом консультация с широким активом требуют от нас анализа новых предложений. С другой стороны постоянное информационно-пропагандистское
обеспечение нашей деятельности позволяет оценить значение дальнейших направлений развития. Повседневная практика показывает, что новая модель
организационной деятельности позволяет оценить значение новых предложений. Задача организации, в особенности же постоянное информационно-пропагандистское
обеспечение нашей деятельности обеспечивает широкому кругу (специалистов) участие в формировании соответствующий условий активизации.

\begin{define}
Грань, у которой все стороны граничные, называется внешней.
\end{define}

\begin{define}
Замкнутый терм - терм без свободных переменных.
\end{define}

\end{document}